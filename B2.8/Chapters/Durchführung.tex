\section{Durchführung}
	Um die Versetzungen in LiF zu untersuchen wurden 2 verschiedene Kristalline Proben zur Verfügung gestellt, eine getemperte, welche 48 Stunden lang auf 650°C erhitzt wurde um möglichst viele Fehlstellen der Kristallstruktur
	aus zu heilen, und eine zweite ungetemperte als Vergleich.\\
	Beide Proben wurden zunächst für 15 Minuten in eine Polierlösung gegeben, anschließend in Ethanol gespült, um die Lösungen nicht zu mischen und damit zu verunreinigen, 
	dann für 5 Minuten in die Ätzlösung gegeben und zuletzt erneut mit Ethanol gespült um den Ätzvorgang zu stoppen.\\
	Im Anschluss an die Präparation wurden die Proben auf einem Objektträger platziert und unter dem Mikroskop betrachtet. Dafür wurden bei beiden Proben 
	Aufnahmen der Oberfläche an verschiedene Positionen gemacht um die Ätzgrübchendichte der jeweiligen Probe bestimmen zu können. Für die getemperte Probe wurde 
	zusätzlich noch eine Aufnahme einer Kleinwinkelkorngrenze gemacht um daraus den Winkel bestimmen zu können. Die notwendigen Messungen der Oberfläche für jede dieser
	Messungen wurde dabei von der Software des Mikroskops übernommen, sodass keine Maßstabs Messung notwendig war.\\
	Nach erster Inspektion der Getemperten Probe wurden 3 Stellen mit möglichst wenigen bereits vorhandenen Ätzgrübchen gewählt um dort anschließend einen guten Kontrast für die Nadeleindrücke zu erhalten.
	Die Nadeleindrücke wurden dabei manuell durch leichtes aufdrücken der Nadel bei möglichst senkrechter Neigung erzeugt, die Probe anschließend erneut für etwas 5 Minuten geätzt, gespült und die Druckstellen unter
	dem Mikroskop betrachtet und die Länger der Arme ausgemessen.
	Im Anschluss wurde die Probe in eine Presse eingespannt und mit 2,716 kg für etwa 2 Minuten belastet und daraufhin noch einmal geätzt und im Mikroskop betrachtet um Aufnahmen der veränderten Eindruckstellen zu machen. 