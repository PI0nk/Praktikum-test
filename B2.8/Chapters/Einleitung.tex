\section{Einleitung}
Dieser Versuch beschäftigt sich mit Kristallfehlern bzw. sogenannten Versetzungen in LiF (Lithiumfluorid). Um diese
näher zu untersuchen, lassen sich eine die Eigenschaften solcher Versetzungen nutzen um Strukturfehler innerhalb der Probe
bzw. Fehler die sich über eine Ebene des Kristalls erstrecken, an der Oberfläche sichtbar zu machen und diese zu untersuchen.
Eine entscheidende Eigenschaften für diesen Versuch ist die erhöhte Ätzbarkeit von Fehlstellen gegenüber der Perfekten Kristallstruktur.
Es wird außerdem das Verhalten solcher Fehlstellen unter Temperatur und Krafteinwirkungen beobachtet durch vergleich verschieden getemperten Proben
und Proben vor und nach Krafteinwirkungen auf je eine Fläche.
