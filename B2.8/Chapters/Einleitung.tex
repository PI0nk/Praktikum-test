\section{Einleitung}
Dieser Versuch beschäftigt sich mit Kristallfehlern bzw. sogenannten Versetzungen in LiF (Lithiumfluorid). Um diese
näher zu untersuchen, lassen sich eine Reihe physikalischer Eigenschaften nutzen, mithilfe derer Rückschlüsse auf die Kristallstruktur des
Festkörpers möglich sind.\\
Die beiden in diesem Versuch genutzten Effekte sind zum Einen die stärkere Ätzbarkeit der Fehlstellen gegenüber dem perfekten Kristall und zum Anderen
die Reflexion von polarisiertem Licht an eben diesen Versetzungen.