\section{Einleitung}
Dieser Versuch beschäftigt sich mit Kristallfehlern bzw. sogenannten Versetzungen in LiF (Lithiumflourid). Um diese
näher zu untersuchen lassen sich eine Reihe von physikalischen Eigenschaften betrachten die Rückschlüsse auf die Kristallstruktur des
Festkörpers zulassen.\\
Die beiden Hauptaspekte die in diesem Versuch genutzt werden ist zum einen die Ätzbarkeit der Fehlstellen gegenüber dem Perfekten Kristall und zum anderen
die Reflexion von polarisiertem Licht an den Versetzungen.