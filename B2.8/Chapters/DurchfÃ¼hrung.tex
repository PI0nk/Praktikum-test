\section{Durchführung}
	Wir erhalten zwei kleine LiF - Proben (Abmessungen ca. 15 \texttimes 3 \texttimes 3 mm$^3$). Eine der Proben wurde für 48 Stunden 
	bei 650$^{\circ}$c getempert, die zweite Probe ist ungetempert. Beide Proben werden zunächst der Reihe nach erst für 15 Minuten in eine
	Polierlösung gegeben, in Ethanol gespült, dann für weitere 5 Minuten in eine Ätzlösung gegeben und zuletzt erneut in Ethanol gespült.
	Mithilfe des Mikroskops machen wir Aufnahmen der Kristalloberflächen zur Bestimmung der Ätzgrübchendichte beider Proben. Ebenso nehmen wir
	Bilder an einer geeigneten Stelle auf, um den Winkel einer Kleinwinkelkorngrenze in der getemperten Probe ermitteln zu können.
	\\
	Nachdem wir eine Seite mit relativ wenigen Fehlstellen auf der getemperten Probe identifizieren, machen wir auf dieser Seite insgesamt 3
	Nadeleindrücke, von denen wir mit dem Mikroskop nach erneutem Ätzen der Probe Aufnahmen anfertigen. Der Kristall wird im Anschluss für 2
	Minuten in die Druckapparatur eingespannt und mit !!!!GEWICHT!!!! Kg belastet. Die Probe wird daraufhin erneut geätzt und es werden weitere
	Bilder der Nadeleindruckstellen mit dem Mikroskop aufgenommen.