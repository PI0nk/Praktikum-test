	Wir erhalten zwei kleine LiF - Proben (Abmessungen ca. 15 \texttimes 3 \texttimes 3 mm$^3$). Eine der Proben wurde f�r 48 Stunden 
	bei 650$^{\circ}$c getempert, die zweite Probe ist ungetempert. Beide Proben werden zun�chst der Reihe nach erst f�r 15 Minuten in eine
	Polierl�sung gegeben, in Ethanol gesp�lt, dann f�r weitere 5 Minuten in eine �tzl�sung gegeben und zuletzt erneut in Ethanol gesp�lt.
	Mithilfe des Mikroskops machen wir Aufnahmen der Kristalloberfl�chen zur Bestimmung der �tzgr�bchendichte beider Proben. Ebenso nehmen wir
	Bilder an einer geeigneten Stelle auf, um den Winkel einer Kleinwinkelkorngrenze in der getemperten Probe ermitteln zu k�nnen.
	\\
	Nachdem wir eine Seite mit relativ wenigen Fehlstellen auf der getemperten Probe identifizieren, machen wir auf dieser Seite insgesamt 3
	Nadeleindr�cke, von denen wir mit dem Mikroskop nach erneutem �tzen der Probe Aufnahmen anfertigen. Der Kristall wird im Anschluss f�r 2
	Minuten in die Druckapparatur eingespannt und mit !!!!GEWICHT!!!! Kg belastet. Die Probe wird daraufhin erneut ge�tzt und es werden weitere
	Bilder der Nadeleindruckstellen mit dem Mikroskop aufgenommen.