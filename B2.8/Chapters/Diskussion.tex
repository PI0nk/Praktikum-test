\section{Diskussion}
Unsere Untersuchung der Ätzgrübchendichte hat wie erwartet gezeigt, dass das Tempern einer Probe die Anzahl der Kristalldefekte verringert.
Die Ätzgrübchendichte der getemperten Probe ist ungefähr drei Zehnerpotenzen geringer als die der ungetemperten Probe.\\
Der Winkel der Kleinwinkelkorngrenze ist mit ca. einem Tausendstel Grad ebenso den Erwartungen entsprechend enorm klein, sodass die verwendete
Näherung zur Winkelbestimmung gerechtfertigt ist.\\
Die Bestimmung des Wachstums der Nadeleindrücke erwies sich hingegen als schwierig, da die Abmessungen der infolge der Belastung gewachsenen Nadeleindruckstellen
zwischen den hinzugekommenen Oberflächendefekten kaum auszumachen sind. Zudem fiel das Wachstum kleiner aus, als erwartet.