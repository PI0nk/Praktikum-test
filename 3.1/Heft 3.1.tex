\documentclass{article}

\usepackage[ngerman]{babel}
\usepackage{pdfpages}
\usepackage{amssymb}
\usepackage{amsmath}
\usepackage{graphics}
\usepackage{graphicx}
\usepackage{geometry}
\usepackage{float}
\usepackage{subfigure}
 
\geometry{
 a4paper,
 total={170mm,257mm},
 left=25mm,
 top=25mm,
}

\begin{document}

    \title{Praktikum B - Versuch B3.1 Statistik der Kernzerfälle}
    \date{\today}
    \author{Jesco Talies, Timon Danowski, Erik Gasmus}
    \maketitle
    \newpage

    \tableofcontents
    \newpage
    
    \section{Einleitung}
        \subsection{Versuchsaufbau}
            In diesem Versuch wollen wir anhand von Kernzerfällen den Einfluss von Messapparaturen auf
            die Messung demonstrieren, dabei betrachten wir eine Ansammlung instabiler Kerne sowie einen Detektor
            zum Nachweis der Zerfälle. Trotz der Unabhängigkeit dieser Zerfälle lässt sich ab einer ausreichend
            großen Zahl an Zerfällen ein Statistischer Zusammenhang erkennen.
        \subsection{Ziel}
            Ziel dieses Versuchs wird sein die oben erwähnten Verteilungen näher kennen zu lernen und dabei
            auch auf den Einfluss der Messapparatur auf die Verteilung zu berücksichtigen. Dafür soll in der 
            Auswertung auf
            \begin{itemize}
                \item die Anzahl der Zerfälle pro Zeitintervall
                \item die Zeitdifferenz zwischen Zerfällen
            \end{itemize}
            geachtet werden und durch einen $\chi^2$-Test untersucht werden.
    
    \section{Versuchsvorbereitung}
            \subsection{Zufallsvariable}
                Im Zentrum dieses Versuchs steht das Betrachten von sogenannten Zufallsvariablen. Da verschiedene Zufallsexperimente
                sehr verschiedene Ereignisse besitzen, ist es dennoch wichtig eine gemeinsame Darstellung zu finden um mit den
                Ergebnissen von Zufallsexperimenten rechnen zu können. Für eine solche Darstellung müssen die Ergebnisse in reelle
                Zahlen transformiert werden, dafür gibt man in allen Fällen eine eindeutige Abbildung
                $$ X: \omega \rightarrow x( \omega ) ; x( \omega ) \in \mathbb{R} $$
                die jedem Ereignis $\omega$ eines Zufallsexperiments eine eindeutige Zahl zuordnet.
                Diese Funktion X bezeichnet man als Zufallsvariable.
            \subsection{Wahrscheinlichkeitsverteilung}
                Die Wahrscheinlichkeitsverteilung bzw. Wahrscheinlichkeitsfunktion einer Zufallsvariablen X ist diejenige Funktion F(x)
                die angibt mit welcher Wahrscheinlichkeit eine entsprechende Zufallsvariable einen Wert kleiner 
                gleich x annimmt. Dabei gilt für diskrete Zufallsvariablen und deren Wahrscheinlichkeitsverteilung
                $$ F(x) = \sum_{x_i < x}f(x_i)$$
                und im Übergang zu kontinuierlichen Zufallsvariablen wird die Wahrscheinlichkeitsdichte beschrieben
                über 
                $$ F(x) = \int_{-\infty}^xf(y)dy$$
            \subsection{Binomialverteilung}
                Die sogenannte Binomialverteilung ist die wohl wichtigste diskrete Verteilung. Sie ist jedoch nur
                anwendbar auf Zufallsexperimente deren Ereignisse genau zwei mögliche, sich gegenseitig ausschließende
                Ergebnisse liefern können. Für die Wahrscheinlichkeit des Eintretens der Ereignisse gilt
                $$P(A) = p ; P(B) = 1 - p $$
                Führt man das Experiment N-mal durch, wobei die Ergebnisse aller N Durchläufe unabhängig voneinander sind,
                kann man über die Anzahl n, welche beschreibt wie oft Ergebnis A eingetreten ist, die Wahrscheinlichkeit 
                folgendermaßen herleiten:
                Nimmt man an das Ereignis A zunächst n mal auftritt und dann (N-n)-mal nicht, dann ist die Wahrscheinlichkeit
                für genau diese Reihenfolge:
                $$ p \underbrace{\cdot p ... \cdot p}_n \underbrace{\cdot (1-p) \cdot (1-p) ... \cdot (1-p)}_{N-n} = p^n \cdot (1-p)^{N-n} $$
                Da für unser Experiment jedoch die Reihenfolge der Ereignisse egal ist, kann über die Kombinatorik die Anzahl
                möglicher Anordnungen einer solchen Verteilung bestimmt werden. Daraus Folgt:
                $$ P(N,n,p) = {N\choose n}p^n(1-p)^{N-n} $$
                Die Binomialverteilung besitzt den Erwartungswert $m=Np$ und die Varianz $\sigma^2=Np(1-p)$.
            \subsection{Poisson-Verteilung}
                Auch die Poisson-Verteilung ist eine diskrete Verteilung, man erhält sie, wenn man die Binomialverteilung
                im Grenzwert $N \rightarrow \infty und p \rightarrow 0$ unter der Nebenbedingung, dass das Produkt $Np =
                \lambda$ konstant ist, betrachtet. Dort erhält man die Wahrscheinlichkeitsfunktion über:
                $$ \lim_{N\rightarrow\infty} P(n) = \lim_{N\rightarrow\infty}{N\choose n}p^n(1-p)^{N-n}=\frac{\lambda^n\cdot e^{-\lambda}}{n!} = P(n,\lambda)$$
                Die Poisson-Verteilung hat den Mittelwert $m = \lambda$  und die Varianz $\sigma^2 = \lambda$.
            \subsection{Gaußverteilung}
                Die Gaußverteilung hingegen ist eine kontinuierliche Verteilung, ihre Wahrscheinlichkeitsdichte ist
                gegeben durch
                $$ P(x,\mu, \sigma) = \frac{1}{\sigma\sqrt{2\pi}}\cdot e^{-\frac{1}{2}(\frac{(x-\mu)^2}{\sigma})}$$
                Dabei ist $\mu$ der Mittelwert und $\sigma^2$ die Varianz.
                \newline
                Für die Gaußverteilung bestehen zwei interessante Zusammenhänge.
                Zunächst erhält man die Gaussverteilung durch eine Grenzwertbetrachtung der Poissonverteilung für $\lambda\rightarrow\infty$
                $$\lim_{\lambda\rightarrow\infty} \frac{\lambda^n e^{-\lambda}}{n!} = \frac{1}{\sqrt{2\pi\lambda}}e^{-\frac{(x-\lambda)^2}{2\lambda}} = P(x,\lambda)$$
                \newline
                Zusätzlich folgt zudem nach dem Satz von Moivre-Laplace, dass falls die Poisson-Verteilung gegen die Gaußverteilung
                konvergiert, so konvergiert auch die Binomialverteilung gegen die Gaußverteilung für $N \rightarrow \infty$ und $0 < p < 1$
                $$\lim_{\lambda\rightarrow\infty} {N\choose n}p^n(1-p)^{N-n} = \frac{1}{\sqrt{2\pi Np(1-p)}} e^{-\frac{(x-Np)^2}{2Np(1-p)}} = P(x)$$
                Für diese Näherung existiert eine Faustregel, nach der die Näherung brauchbar ist falls gilt $Np(1-p)\geq 9$.
                Je asymmetrischer die Binomialverteilung ist, desto größer muss N sein, bevor die Gaußverteilung eine gute Näherung liefert.
            \subsection{Varianz}
                Für jede der oben angegebenen Verteilungen lässt sich eine Varianz $\sigma^2$ angeben, diese Varianz ist ein Maß für die Streuung
                einer Zufallsvariablen X um ihren Erwartungs- bzw. Mittelwert $\mu$. Diese lässt sich auf zwei Wege bestimmen, entweder falls der
                tatsächliche Mittelwert bekannt ist über
                $$Var(X) = \int_\omega (X-\mu)^2 dP$$
                als Integral über den Wahrscheinlichkeitsraum $\Omega$, oder falls nur der experimentell bestimmte Mittelwert bekannt ist
                $$Var(X) = \sum_{i\geq 1}(x_i - \mu)^2\cdot p_i$$
                wobei $p_i$ der Wahrscheinlichkeit entspricht mit der X den Wert $x_i$ annimmt.
            \subsection{Zerfallswahrscheinlichkeit}
                Den Zusammenhang zwischen einem Zufallsexperiment und unserem Versuch bildet die Zerfallswahrscheinlichkeit.
                Die Wahrscheinlichkeit $\omega$, dass ein Atomkern unseres instabilen Kerns mit einer Halbwertszeit $ T_{1/2} $ (Hier $^{137}Cs$ mit $ T_{1/2}=30 Jahre $ )
                innerhalb einer bestimmten Zeitspanne $\Delta t \ll T_{1/2}$ zerfällt, beträgt
                $$w = \alpha \Delta t$$
                $\alpha$ ist dabei für das Isotop charakteristische Konstante und wird die Zerfallswahrscheinlichkeit genannt.
                Diese Gleichung ist jedoch nur gültig für infinitesimale $\Delta t$, da sonst Wahrscheinlichkeiten > 1 erreicht werden könnten.
                Für beliebige endliche t gilt für die Zerfallswahrscheinlichkeit
                $$\omega(t)=1-e^{-\alpha t}$$
                Mit bekannter Halbwertszeit berechnet sich $\alpha$ folgendermaßen
                $$ \omega(T_{1/2}) = \frac{1}{2} \Leftrightarrow e^{-\alpha T_{1/2}} = \frac{1}{2} \Leftrightarrow \alpha = \frac{ln(1/2)}{T_{1/2}}$$
                In unserem Fall mit dem Isotop $^137Cs$ ist $\alpha = 7,3 \cdot 10^{-6} s^{-1}$.
                Hier ist bereits der erste Zusammenhang zu unseren Wahrscheinlichkeitsverteilungen zu erkennen Es handelt sich um ein
                Zufallsexperiment bei dem die zwei sich ausschließenden Ereignisse das Zerfallen bzw. das nicht Zerfallen eines Kerns sind.
                Diese Zerfälle sind von einander Unabhängigkeit wodurch unser Experiment als Binominalverteilt angenommen werden kann.
                Für viele Kerne ist es nun sinnvoll einen Ausdruck zu finden für die Wahrscheinlichkeit $\omega(N,n,t)$, dass von N
                Kernen, während der Zeit t, n zerfallen. Die Anzahl der Möglichen  Zerfälle ist über den Binomialkoeffizienten gegeben.
                $$ \omega(N,n,t) = {N\choose n}\cdot(1-e^{-\alpha t})^n e^{-\alpha(N-n)t}$$
                Mit der Abkürzung $p = 1 - e^{-\alpha t}$
                erhalten wir
                $$P(N,n,p) = {N\choose n}\cdot p^n (1-p)^{N-n}$$
                eine Wahrscheinlichkeitsfunktion einer Binomialverteilung.
                Da es sich bei unserem Experiment um eine sehr große Anzahl von Kernen (Größenordnung von $10^{20}$) 
                und Wahrscheinlichkeiten eines Zerfalls von $\alpha \ll 1$ (hier $7,3 \cdot 10^{-6} s^{-1}$) handelt, können wir für
                große Proben unsere Binomialverteilung als Poisson-Verteilung nähern mit
                $$P(n) = \frac{\lambda^n}{n!} e^{-\lambda}$$
                wobei n der registrierten Zählrate entspricht und $\lambda$ dem Erwartungswert der Zerfälle im Zeitraum t.
                Für große $\lambda$ d.h. $\arrowvert n - \lambda \arrowvert \ll \lambda$ (große Messezeiten) geht die Poisson-Verteilung
                sogar in eine spezielle Gaußsche Normalverteilung über
                $$\Phi(n) = \frac{1}{\sqrt{2\pi}\sigma} e^{-\frac{(n-\lambda)^2}{2\sigma^2}}$$
            \subsection{Intervallverteilung}
                Eine weitere Verteilung ist die Intervallverteilung. Bei dieser Verteilung wird im Gegensatz zu den bisherigen Verteilungen, welche die
                Wahrscheinlichkeit für das Zerfallen von Kernen in einer festen Zeitspanne t beschrieben haben, die Zeitdifferenz zwischen
                zwei Zerfällen betrachtet. Betrachtet wird also mit welcher Wahrscheinlichkeit die Zeitdifferenz zwischen zwei Zerfällen
                den Wert $\Delta t$ annimmt. Die Wahrscheinlichkeit die durch diese Betrachtung beschrieben wird setzt sich aus zwei Einzelwahrscheinlichkeiten
                zusammen.
                \begin{itemize}
                    \item $W_1$: In der Zeit $\Delta t$ darf kein Zerfall stattfinden, die Wahrscheinlichkeit entspricht nach der Poisson-Verteilung
                    mit $n = 0 : W_1 = e^{-\lambda}$.
                    \item $W_2$: Nach der Zeit $\Delta t$ muss innerhalb von dt ein Zerfall Stattfinden. Die Wahrscheinlichkeit ist $W_2 = a\cdot dt$.
                \end{itemize}
                Daraus folgt:
                $$ W = W_1 \cdot W_2 = e^{-\lambda} a dt$$
                Um auf die Dichtefunktion zu kommen muss nun lediglich abgeleitet werden:
                $$P = e^{-at}a$$
                Die Dichtefunktion dafür, dass zwischen den beiden betrachteten Ereignissen noch n weitere liegen folgt analog und ergibt sich zu:
                $$P_n = a \frac{(at)^n}{n!}e^{-at}$$
                Diese Verteilung nennt man Intervallverteilung.
            \subsection{Zählrohr}
                Zum Nachweis bzw. zur Messung der in den vorherigen Abschnitten angesprochenen Zerfälle wird ein Geiger-Müller Zählrohr
                verwendet. Es besteht aus einem Gas gefüllten Zylinderkondensator mit einem dünnen Anodendraht. Tritt nun ionisierende
                Strahlung in das Rohr ein, die bei einem Radioaktiven Zerfall entsteht (Hier $\beta^-$ $Z(A,Z) \rightarrow Z(A,Z+1) + e^- + \nu^-)$)
                löst diese Elektronen aus den Hüllen des Gases. Die nun freien Elektronen bewegen sich durch das Elektrische Feld zur Anode
                und lösen auf ihrem Weg weitere Elektronen raus. Zusätzlich regen sich die Ionisierten Gasteilchen unter Emission von Photonen ab
                die wiederum Elektronen raus lösen, sodass eine großräumige Entladung im Zählrohr stattfindet, unabhängig von der ursprünglichen Ionisation.
                Die durch die oben beschriebenen Effekte entstehende Elektronenlawine liefert dabei unser Messbares Signal.\newline
                Die Entladung über diese Kettenreaktion endet erst, wenn die Radial nach außen wandernde Wolke ionisierter Gasteilchen die 
                Feldstärke genügend Verringert hat. Diese relativ lange Zeit bezeichnet man als Totzeit, da währenddessen keine zusätzlichen
                Zerfälle detektiert werden können.
                Um Mehrfachentladungen durch weitere Ionisierungseffekte von Gasatomen an der Kathoden Oberfläche zu vermeiden, wird das Zählrohr
                mit einem hochohmigen Widerstand im Bereich $10^8 \Omega$ zwischen Hochspannungsversorgung und Anode ausgestattet. Der Widerstand
                führt dazu, dass die Hochspannung nach einer gewissen Zeit zu gering ist, um weitere Entladungen auszulösen.
            \subsection{Totzeit}
                Die in vorherigen Absatz bereits erwähnte Totzeit bezeichnet die Zeit nach einem Nachweis in dem kein weiteres Teilchen nachgewiesen
                werden kann. Diese Totzeit wirkt sich dadurch auf die Zählrate unserer Messung aus, die gemessene Zählrate a' ist kleiner als
                die Tatsächliche Zählrate a. Mit der Totzeit $\tau$ ergibt sich:
                $$ a' (1+ a\tau) = a \Leftrightarrow a = \frac{1}{1-a'\tau}$$
                Um den wahren Mittelwert vom Gemessenen zu erhalten gilt:
                $$ m = aT = a\frac{m'}{a'} = \frac{m'}{1-a'\tau}$$
            \subsection{Zwei-Präparat-Methode}
                Um die Totzeit zu messen lässt sich die Zwei-Präparat-Methode verwenden, dabei wird bei einer Festen Zählrohrspannung
                die Zählrate von Zwei Präparaten erst jeweils separat und dann gleichzeitig gemessen. Zudem wird die Untergrundzählrate
                bestimmt. Die gemessenen Zählraten $z_1',z_2',z_{12}'$ und die Untergrundzählrate bestimmen unsere wahren Zählraten $z_1,z_2,z_{12}$,
                die durch die Präparate zustande kommenden Zählraten seien $p_1,p_2,p_{12}$
                $$ z_1 = p_1 + u  ;  z_2 = p_2 + u  ;  z_{12} = p_{12} + u $$
                Mit der Totzeit $\tau$ gilt:
                $$ z_i = \frac{z_i'}{1-z_i'\tau}$$
                Die Idee dabei ist es die Additivität der Präparats Zählraten auszunutzen
                $$ p_{12} = p_1 + p_2$$
                Das daraus resultierende Gleichungssystem lässt sich durch einsetzten der vorherigen Gleichung in die jeweiligen Gleichungen für $z_i$
                lösen:
                $$z_{12} - u = z_1 - u + z_2 - u \Leftrightarrow z_{12} + u = z_1 + z_2$$
                $$\frac{z_{12}'}{1-z_{12}'\tau} + \frac{u'}{1-u'\tau} = \frac{z'}{1-z_1'\tau} + \frac{z_2'}{1-z_2'\tau}$$
                Löst man diese Gleichung nach $\tau$ auf ergibt sich:
                $$(\underbrace{u'z_{12}'z_2'-z_1'z_12'z_2'+u'z_12'z_1'-u'z_1'z_2'}_=A)\tau^2 + (\underbrace{-2z_{12}'u'+2z_1'z_2'}_=B)\tau + \underbrace{z_{12}'-z_1'+u'-z_2'}_=C = 0$$
                $$ \tau = \frac{-B \pm \sqrt{B^2 - 4AC}}{2A}$$
            \subsection{Der $\chi^2$-Anpassungstest}
                
    
    Folgende Fragen sind notwendig für das Verständnis des Versuchs und dienen als Vorbereitung für diesen.

        \begin{enumerate}
            \item Wie misst man die Totzeit mit der Zwei-Präparate-Methode?
                \begin{itemize}
                    \item Bei \emph{der Zwei-Präparate-Methode} wird zunächst die Zählrate des ersten Präparats bei fester Zählrohrspannung gemessen, dann die des zweiten Präparates
                    und zuletzt die der beiden zusammen. Zudem wird die Untergrundzählrate bestimmt, d.h. das Hintergrundrauschen, welches durch die Umgebung entsteht.
                    - TODO - Formeln?
                \end{itemize} 
            \item Wie wirkt sich die Totzeit auf eine gemessene Verteilung aus?
                \begin{itemize}
                    \item Die Totzeit führt zu einer schmaleren gemessenen Verteilung bzw. zu einer Verkleinerung von $\chi^2$. \newline
                    Die gemessene Zählrate $a'$ ist durch die aufgrund der Totzeit fehlenden Messwerte kleiner als die Tatsächliche Zählrate a.
                    Für jedes Teilchen muss man daher $a\cdot\tau$ addieren
                    $$a=a'(1+a\tau)$$
                    für den gemessenen $m'$ und tatsächlichen Mittelwert $m$ gilt dann
                    $$m = aT = a\frac{m'}{a'}=\frac{m'}{1-a'\tau}$$
                    Da die Messwerte aus einer Poisson-Verteilung stammen gilt außerdem $\sigma^2=m$
                    Insgesamt folgt also
                    $$\sigma'^2 = m' \cdot (1-\tau a') bzw \sigma^2=\sigma'^2 \cdot (\frac{a}{a'})^2$$
                    Da gilt $\frac{a}{a'} > 1$. Ist also wie bereits erwähnt die gemessene Verteilung schmaler als die wahre.
                \end{itemize} 
            \item Was ist eine $\chi^2$-Verteilung und wie funktioniert ein $\chi^2$-Test?
                \begin{itemize}
                    \item \emph{Die $\chi^2$-Verteilung} ist eine stetige Wahrscheinlichkeitsverteilung über die menge der positiven reellen Zahlen über einen einzigen Parameter,
                    die Anzahl der Freiheitsgrade n. Sie kann aus der Normalverteilung abgeleitet werden indem man n Zufallsvariablen $Z_i$ betrachtet die unabhängig und standardnormalverteilt sind.
                    Das Quadrat einer standardnormalverteilten Zufallsvariable $Z - N(0,1)$ folgt einer $\chi^2$-Verteilung mit Freiheitsgrad
                    $$Z^2~\chi^2(1)$$
                    Für jede standardnormalverteilte Zufallsgröße gilt, dass sie den Erwartungswert 0 hat und das Quadrat der selbigen den Erwartungswert 1. Bei f 
                    dieser Größen addieren sich die Erwartungswerte zu f. Betrachtet man nun die Wahrscheinlichkeitsdichte Funktion der $\chi^2$-Verteilung kann man sich nun einen
                    Ablehnungsbereich festlegen. Für eine Irrtumswahrscheinlichkeit von $5\%$ bestimmte man folgendes Integral bzw. $\chi_{min}^2$ und $\chi_{max}^2$
                    $$\int_{x=0}^{\chi_{min}^2}f(x,f)dx = 0.025$$
                    $$\int_{x=0}^{\chi_{max}^2}f(x,f)dx = 0.975$$
                    \item \emph{Mit dem $\chi^2$-Test} soll geprüft werden ob zwei Verteilungen \newline
                    übereinstimmen bzw. ob die Verteilung einer Zufallsvariable einer hypothetischen Verteilung entspricht.   
                \end{itemize} 
            \item Warum können wir mit dem $\chi^2$-Test Hypothesen ablehnen oder nicht ablehnen, aber niemals annehmen?
                \begin{itemize}
                    \item Da es sich bei der $\chi^2$-Verteilung um das Quadrat der ursprünglichen Verteilung handelt kann man zwar Aussagen über die Abweichung treffen,
                    jedoch keine aussagen über die Annahme der ursprünglichen Hypothese treffen, da diese auf der nicht quadrierten Verteilung beruht.
                    Daher kann man lediglich Ablehnen bei zu großer bzw. zu kleiner Abweichung, jedoch nie annehmen.
                \end{itemize} 
            \item Das $\chi^2$ ist ein Maß für die Abweichung der Daten von der Hypothese. Warum ist es trotzdem wichtig, auch bei einem kleinen $\chi^2$ die 
            Hypothese abzulehnen? Bedeutet ein kleines $\chi^2$ nicht, dass die Daten sehr gut zur Hypothese passen?
                \begin{itemize}
                    \item Ist die Differenz $\chi_{max}^2 - \chi_{min}^2$ zu klein besteht das Risiko eines Messfehlers, da man bei gemessenen Zufallsgrößen stets
                    von einem Mindestmaß an Streuung ausgeht bzw. ausgehen muss. Findet sich eine zu kleine Streuung sollte die Messung ebenfalls in Frage gestellt werden
                    und ggf. sogar der Messvorgang überprüft oder verändert werden.
                \end{itemize} 
            
        \end{enumerate}
    \section{Versuchsdurchführung}
        Um die Verteilungen zu ermitteln, werden zunächst die Zeitpunkte relativ zum Messstart, an denen ein Zerfall detektiert wurde
        von einem Computer in ein Textdokument geschrieben, welche in der Auswertung weiter aufbereitet werden.
        Für die Auswertung liegen Daten aus den Folgenden Messungen vor:
        \begin{itemize}
            \item 45 Minuten (2700s) mit Quelle Nr. 7 bei 500V
            \item 45 Minuten mit Quelle Nr. 7 bei 600V
            \item 45 Minuten mit Quelle Nr. 6 und Nr. 7 bei 500V
        \end{itemize}
        Zur Bestimmung der Totzeit über die Zwei-Präparate-Methode liegen die Zählraten für die Messungen
        \begin{itemize}
            \item 5 Minuten mit Quelle Nr. 6 bei 500, 600 und 700V
            \item 5 Minuten mit Quelle Nr. 7 bei 500, 600 und 700V
            \item 5 Minuten mit Quelle Nr. 6 und Nr. 7 bei 500, 600 und 700V
            \item 5 Minuten nur Untergrund bei 500, 600 und 700V
        \end{itemize}
    \section{Auswertung}
        \subsection{Poisson-Verteilung}
            Im ersten Teil der Auswertung befassen wir uns mit der Extraktion von Poisson-Verteilungen aus den Aufgenommenen
            Daten. Um eine Wahrscheinlichkeitsverteilung aus den Messwerten zu extrahieren mussten zunächst die 
            Daten ausgewertet werden in Hinsicht auf die Anzahl der Zerfälle pro Zeitintervall. Dabei ist es wichtig 
            die richtige Länge des Zeitintervalls zu wählen, da bei zu kleinen Zeitintervallen keine statistisch verwertbare
            Menge an Ereignissen vor liegt , jedoch gilt der Übergang von der Binomial- zur Poisson-Verteilung nur im Grenzfall
            $$P \rightarrow 0, N \rightarrow \infty$$
            Für unsere Plots haben wir eine Zeitauflösung von 200 Millisekunden gewählt. \newline
            Aus den resultierenden Histogrammen lies sich nun über Fitten der Wahrscheinlichkeitsfunktion
            der Poisson-Verteilung
            $$P(n,\lambda) = N \cdot \frac{\lambda^n}{n!} \cdot e^{-\lambda}$$
            N bestimmt sich dabei als Summe aller Balkenhöhen, hier $N = 13499$.\newline 
            der Fit-Parameter $\lambda$ und somit direkt auch der Erwartungswert und die Varianzen bestimmt.
            Aus den Folgenden Plots erhielten wir folgende Werte:
            \begin{center}
                \begin{tabular}{ |c|c|c| }
                    \hline
                    Quelle & Erwartungswert $\lambda$ &  Varianz $\sigma^2$ \\
                    \hline
                    Quelle 7 500V & $10.99 \pm  0,299$&   $10.99 \pm  0,299$   \\
                    Quelle 7 600V & $12.39 \pm 0,189$&    $12.39 \pm 0,189$     \\
                    Quelle 6 und 7 500V & $14.83 \pm 0,545$&   $14.83 \pm 0,545$ \\
                    \hline
                \end{tabular}
            \end{center}


            \begin{figure}[H]  
                \centering 
                 \subfigure[Quelle 7 500V]{\includegraphics[width=0.3\textwidth]{Pics/Poisson Quelle 7 500V.png}}\quad 
                 \subfigure[Quelle 7 600V]{\includegraphics[width=0.3\textwidth]{Pics/Poisson Quelle 7 600V.png}}\quad
                 \subfigure[Quelle 6 und 7 500V]{\includegraphics[width=0.3\textwidth]{Pics/Poisson Quelle 6 und 7 500V.png}}
                 \caption{Gefittete Poisson-Verteilungen} 
             \end{figure}



        \subsection{Gaussverteilung}
            Wie wir bereits in der Theoretischen Vorbereitung angesprochen haben kann man ein solches Zerfallsexperiment
            nicht nur als Poissonverteilt nähern, man kann es sogar als Gaußverteilt betrachten. Die Gaußverteilung
            für unsere gemessenen Daten müssen wir in unseren Fall jedoch nicht fitten sonder können direkt alle Parameter
            berechnen. Für die Verteilung gilt:
            $$ P(x) = \frac{1}{\sqrt{2\pi m}} \cdot F \cdot \sqrt{n} e^{-\frac{(x-m)^2}{2m/n}}$$
            mit
            $$ m = z'\cdot\frac{\Delta t}{n} $$
            wobei m unseren Mittelwert, z' unsere gemessene Zählrate, F den Normierungsfaktor und n die Anzahl der Histogrammbalken
            angibt die zu einem zusammengefasst werden.
            Daraus folgt
            \begin{center}
                \begin{tabular}{|c|c|c|c|c|c|}
                    \hline
                    Quelle(n) & n & z' & $\Delta t$ & m & F \\
                    \hline
                    Quelle 7 500V & 8 & 5,499 & 5 & 3,437 & 539 \\
                    Quelle 7 600V & 8 & 5,930 & 5 & 3,706 & 539 \\
                    Quelle 6 und 7 500V & 8 & 8,842 & 5 & 5,526 & 539 \\
                    \hline 

                \end{tabular}
            \end{center}

            \begin{figure}[H]  
                \centering 
                 \subfigure[Quelle 7 500V]{\includegraphics[width=0.3\textwidth]{Pics/Gauss Quelle 7 500V.png}}\quad 
                 \subfigure[Quelle 7 600V]{\includegraphics[width=0.3\textwidth]{Pics/Gauss Quelle 7 600V.png}}\quad
                 \subfigure[Quelle 6 und 7 500V]{\includegraphics[width=0.3\textwidth]{Pics/Gauss Quelle 6 und 7 500V.png}}
                 \caption{Geplottete Gaußverteilungen über Gemessener Verteilung} 
             \end{figure}

            Wie aus den Graphen klar hervorgeht, liegt unsere Theoretische Verteilung in den Messungen der einzelnen
            Quellen. Diese Diskrepanz lässt sich über die Untergrundstrahlung erklären, durch welche 
            mehr Ereignisse registriert werden, als von unserer Quelle abgestrahlt werden. Wir hätten eigentlich eine 
            theoretische Verteilung oberhalb der Gemessenen erwartet, da durch die Totzeit eigentlich Ereignisse verschwinden
            sollten, sollte die Aktivität unserer Quelle jedoch nicht hoch genug gewesen sein um eine Signifikante Menge 
            der Signale durch Totzeit zu verlieren könnten einige Signale aus der Untergrundstrahlung aufgezeichnet worden sein.
            Bei der Messung beider Quellen ist das oben bereits erwähnte Phänomen zu erkennen, bei dem einige der Theoretisch
            zu erwartenden Ereignisse durch die Totzeit verschluckt werden und somit die Gemessene Verteilung unterhalb der Theoretischen
            liegt.

        \subsection{Intervallverteilung}
            In diesem Teil des Versuchs sollte eine Intervallverteilung aus den Messdaten der Messung zu Quelle 7
            extrahiert. Dazu untersuchten wir die zeitlichen Abstände zwischen 1,2 bzw. 3 Zerfällen in Zeitintervalle der Länge 
            $\Delta t$ = 0,001s und plotteten die jeweiligen Intervallverteilungen der Messdaten für n=0,1,2.
            Des weiteren bestimmten wir die Totzeit, indem wir für den Fall n=0 einen Fit mit folgender Funktion vornahmen:
            $$Z(t) = A\cdot\frac{a'}{1-a'\tau}\cdot e^{-\frac{a'}{1-a'\tau}\cdot t}$$
            Dabei ist A die Anzahl der gemessenen Ereignisse, a' die Zählrate und $\tau$ die Totzeit.
             \begin{figure}[H]
                 \centering
                 \includegraphics[width=0.5\textwidth]{Intervall all.png}
                 \label{fig:Intervall plot n=0,1,2}
             \end{figure}

            \begin{figure}[H]
                \centering
                \includegraphics[width=0.5\textwidth]{Intervall Quelle 7 600V.png}
                \label{fig:Intervall fit, Quelle 7 600V}
            \end{figure}

            Durch den Fit erhielten wir eine Totzeit von 
            $$\tau = 6,723 \pm 0,4008$$
        
        \subsection{$\chi^2$-Test}
            In diesem Versuchsteil ging es um die Untersuchung dreier Hypothesen mithilfe eines $\chi^2$-Tests.
            \begin{enumerate}
                \item Die Präparatstärke ist konstant im betrachteten Zeitraum und gleicht dem Mittelwert der 51 Messwerte.
                \item Die Präparatstärke ist konstant im betrachteten Zeitraum und gleicht dem Mittelwert der 51 Messwerte minus 10%.
                \item Die Präparatstärke nimmt im betrachteten Zeitraum linear mit der Zeit ab. Die Anfangszählrate ist der Mittelwert
                und der Abfall von einer Messung zur anderen sei 1.
            \end{enumerate}
            Um diese Hypothesen zu Untersuchen plotteten wir die Extrahierten Messdaten als Histogramm und stellten die Hypothesen Graphisch dar.
            \begin{figure}[H]
                \centering
                \includegraphics[width=0.5\textwidth]{CHI Quelle 6 und 7 500V.png}
                \label{fig:Messdaten und Hypothesen}
            \end{figure}

        \subsection{Totzeit}
            Zur Bestimmung der Totzeit wurden jeweils bei 500, 600 und 700V die Zählraten von Quelle 6,7 und beiden gleichzeitig
            sowie die des Untergrunds aufgenommen.
            \begin{center}
                \begin{tabular}{|c|c|c|c|}
                    \hline
                    Quelle & 500V & 600V & 700V \\
                    \hline
                    6 & 16735 & 17840 & 18343 \\
                    7 & 14848 & 16010 & 16477 \\
                    6+7 & 23873 & 26897 & 28711 \\
                    Untergrund & 2335 & 2432 & 2442 \\
                    \hline
                \end{tabular}
            \end{center}

            \begin{center}
                \begin{tabular}{|c|c|c|c|}
                    \hline
                    Totzeit & 500V & 600V & 700V \\
                    \hline
                    $\tau_+$[ms] & 0.

                \end{tabular}
            \end{center}

    \section{Diskussion}

\end{document}