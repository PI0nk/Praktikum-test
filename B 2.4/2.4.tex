\documentclass{article}

\usepackage[ngerman]{babel}
\usepackage{pdfpages}
\usepackage{amssymb}
\usepackage{amsmath}
\usepackage{mathtools}
\usepackage{graphics}
\usepackage{graphicx}
\usepackage{geometry}
\usepackage{float}
\usepackage{color,soul}

\geometry{
 a4paper,
 total={170mm,257mm},
 left=25mm,
 top=25mm, 
}


\begin{document} 
 
\thispagestyle{empty}
\vspace*{\fill}
\begin{center}
	\Huge
	\textbf{Universität zu Köln}\\
	\LARGE
	\textbf{Institut für Festkörperphysik}\\
	\vspace{2cm}
	\textbf{Versuchsprotokoll}\\  
	\vspace{0.5cm}
	\large
	\textbf{B2.4: Magnetisierungskurve eines Ferrits}\\
	\normalsize
	\vspace{2cm}
	\begin{tabular}{r l}
		Autoren: 	& Jesco Talies$^1$\\
					& Timon Danowski$^2$\\
                    & Erik Gassmus$^3$\\
		Durchgefuehrt am:	& 10.06.2021\\
		Betreuer:	& Philipp Warzanowski
	\end{tabular}
\end{center}
\vfill\footnotesize
$^1$ jtalies@smail.uni-koeln.de, Matrikel-Nr.: 7348338\\
$^2$ tdanowsk@smail.uni-koeln.de, Matrikel-Nr.: 7348629\\
$^3$ egassmus@smail.uni-koeln.de, Matrikel-Nr.: 1111111\\
\normalsize

\newpage
\thispagestyle{empty}
\tableofcontents
\clearpage
\setcounter{page}{1}
\section{Einleitung}
    In vielen Legierungen bildet sich zusätzlich zu der Gitterstruktur des Festkörpers eine übergeordnete
    Struktur, die sogenannte Überstruktur. Sie lässt sich in vergleichsweise Makroskopischen Systemen
    über die Minimierung der Energie erreichen und ist häufig beeinflusst durch Fehlstellen und Deformationen.
    Diese Überstrukturen lassen sich beeinflussen bzw. erzeugen, sie treten nur unterhalb einer kritischen
    Temperatur auf, sodass sich durch gezieltes Erhitzen und Abkühlen eines Systems, Proben mit mehr
    oder Weniger Ordnung erzeugen lassen, sodass im resultierenden Spektrum die Unterschiede zu erkennen sind.
    Im folgenden Versuch werden wir uns genau dieses Phänomen zu nutze machen, indem drei verschieden geordnete
    Proben miteinander vergleichen werden. Dazu wird zunächst die röntgenographische Methode und anschließend
    die restive verwendet.

    \begin{figure}[H]
        \centering
        \includegraphics{images/einleitung_hurensohn.jpg}
        \label{einleitung}
        \caption{Basiszelle Cu$_3$Au}
    \end{figure}  
   
\section{Theoretische Vorbereitung}
    \subsection{Reziprokes Gitter}
        Das reziproke Gitter beschreibt in der Festkörperphysik die Röntgen-, Elektronen-, und Neutronenbeugung
        an Kristallinen strukturen. Es wird häufig in zusammenhang mit den Miller'schen Indizes verwendet
        um die Netebenen $(hkl)$ zu beschreiben. Es bietet sich an diese im Reziproken zu definieren, da die Länge
        eines Vektors der die Position eines Gitterpunkts beschreibt gleich dem Reziproken des Abstands der
        Netzebenen entspricht.
        Aus den Basisvektoren des Punktgitter ($\vec{a_1},\vec{a_2},\vec{a_3}$) ergeben sich über folgende Beziehung
        die Basisvektoren ($\vec{b_1},\vec{b_2},\vec{b_3}$) des Reziproken gitters.
        \begin{align*}
            \vec{b_1} = 2\pi \frac{\vec{a_2}\times \vec{a_3}}{\vec{a_1}\cdot (\vec{a_2}\times \vec{a_3})}
            \\\vec{b_2} = 2\pi \frac{\vec{a_3}\times \vec{a_1}}{\vec{a_1}\cdot (\vec{a_2}\times \vec{a_3})}
            \\\vec{b_3} = 2\pi \frac{\vec{a_1}\times \vec{a_2}}{\vec{a_1}\cdot (\vec{a_2}\times \vec{a_3})}
        \end{align*}
        Über dieses Definition der Basisvektoren lassen sich die Koordinateneines Punktes im reziproken Gitter
        über die Miller'schen indizes $(hkl)$ beschreiben.
        
        \subsubsection*{Bragg Gleichung}
            Die Bragg Gleichung liefert einen Zusammenhang zwischen dem Netzebenenabstand $d_{hkl}$ und dem
            Beugungswinkel $\theta$. Damit dieser Zusammenhang gilt muss jedoch der einfallende und gestreute
            Strahl symetrisch zur reflektierende Netzebene verlaufen. Dann lässt sich der Zusammenhang beschreiben durch
            \begin{equation}
                n\lambda = 2d_{hkl} \sin(\theta)
            \end{equation}
            aus dieser lässt sich die äquivalente Laue Bedingung ableiten, welche aussagt,
            dass ein Röntgenstrahl genau dann gestreut wird, wenn der Beugungsvektor $\vec{k}$ gleich dem
            reziproken Gittervektor ist. 
    
    \subsection{Ordnungsparameter und Phasenübergänge}
            

    \subsection{Überstrukturen}
        
        \subsubsection{$CuZn$}

        \subsubsection{$CuAu$}

        \subsubsection{$Cu_3Au$}

    \subsection{Die röntgenographische Methode}

        \subsubsection{Aufbau eines Röntgendiffraktometers}   

        \subsubsection{Röntgenstrahlung}

        \subsubsection{Intensität der gestreuten Röntgenstrahlung}

        \subsection{Reflexindizierung im Röntgendiffraktogramm}
    
    \subsection{Die resistive Methode}
        
  

%\section{Versuchsafbau}
    \subsection*{Röntgenografische Methode}
    Für die Röntgenografische methode steht im versuch ein Röntgendiffraktometer zur verfügung.
    Dabei sitzt die Probe in der Mitte eines Detektors welcher Röntegenstrahlung detektieren kann.
    Die Probe wird dann mit einer von einer Röntgenröhre erzeugter Röntegenstrahlung bestrahlt, welche
    an den Gitterebenen der Probe reflektiert und auf dem Detektorschirm abgebildet wird.
    \begin{figure}[H]
        \centering
        \begin{subfigure}{.5\textwidth}
        \centering
        \includegraphics[width=.8\linewidth]{images/diffraktometer.png}
        \caption{}
        \label{fig:sub1}
        \end{subfigure}%
        \begin{subfigure}{.5\textwidth}
        \centering
        \includegraphics[width=.6\linewidth]{images/diffraktometer_pic.png}
        \caption{}
        \label{fig:sub2}
        \end{subfigure}
        \caption{a) Schematischer Aufbau eines Röntgendiffraktometers b) Foto des Röntgendiffraktometers}
        \label{fig:test}
    \end{figure}
    \subsection*{Resistives Verfahren}
    Bei dem resistiven Verfahren müssen zunächst die Proben auf einem Stab fixiert werden, anschließen werden
    an der Probe die Leitungen mit Silberpaste leitend befestigt. Anschließend wird der Stab mit der Fixierten
    Probe an einem Schrittmotor Befestigt um die Probe in einen Helium behälter auf variable höhen zu bewegen.
    Beim befestigen der Leiter auf der Probe ist dabei zu achten, dass der abstand zwischen den Kontakten möglichst
    homogen ist und, dass die Kontakte richtig verschaltet sind um die Messung nicht durch die Innenwiderstände
    der Messelektronik zu beeinflussen. Dazu nutzt mann folgenden schaltplan
    \begin{figure}[H]
        \centering
        \begin{subfigure}{.5\textwidth}
        \centering
        \includegraphics[width=.4\linewidth]{images/schaltplan.png}
        \caption{}
        \label{fig:sub11}
        \end{subfigure}%
        \begin{subfigure}{.5\textwidth}
        \centering
        \includegraphics[width=.8\linewidth]{images/aufbaures.png}
        \caption{}
        \label{fig:sub22}
        \end{subfigure} 
        \caption{a) Beschaltung des 4-Punkt messstabes b) Bild des Messaufbaus mit Motor, Proben messtab und Heliumbehälter}
        \label{fig:test1}
    \end{figure}

 

%\section{Durchführung}
    Es werden drei $Cu_3Au$ Proben auf ihre Ordnungsparameter untersucht. Um einen unterschiedlichen Ordnungsparameter
    zu erzeugen, wurden die drei Proben auf über 500 °C ($>T_C = 386 °C$) erhitzt und danach unterschiedlich abgekühlt.
    Um eine vollständige Ordnung zu erzeugen, wird eine der Proben sehr langsam abgekühlt. Für Unordnung wird eine Probe
    in Wasser abgeschreckt. Die teilweise Ordnung wird durch ca. zweistündiges halten bei 370 °C mit anschließendem abkühlen
    erreicht. Von jeder dieser Proben wird nun ein Röntgendiffraktogramm aufgenommen, sowie der elektrische Widerstand 
    mittels der Vierpunktmethode bestimmt. Daraus lassen sich dann die Ordnungsparameter der Proben bestimmen.

\end{document}