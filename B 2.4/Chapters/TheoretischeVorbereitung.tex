\section{Theoretische Vorbereitung}
    \begin{itemize}
        \item für die Sekundärspule ist keine homogene Bewickelung notwendig, warum?
        \item 
    \end{itemize}
    \subsection{Grundlegende Beziehungen}
        \subsubsection{Magnetfelder}
            In der Physik wird das Vektorielle Magnetfeld meist als B-Feld bezeichnet. Es besitzt den Formelbuchstaben
            $\vec{B}$ und wird genauer auch als Magnetische Flussdichte bezeichnet und es wird in der Regel in Tesla angegeben.
            Das B-Feld ist jedoch keine Messbare größe, die zugehörige Messgröße zum B-Feld ist die Magnetische Feldstärke $\vec{H}$.
            Die Magnetische Feldstärke $H$ wird in Ampere pro Meter angegeben ($\frac{A}{m}$) und ist im Vaccum auch Direkt proportional
            zum B-Feld. Dort gilt
            \begin{equation}
                \vec{B} = \mu_0 \vec{H}
            \end{equation}
            $\mu_0$($\eqsim 4\pi ×10^{-7 }\frac{N}{A^2}$) bezeichnet dabei die Magnetische permeabilität von Vaccum.
            In Anwesenheit eines Material mit Magnetisierungs Vektor $M [\frac{A}{m}]$, welcher die dichte der im Material induzierten oder permanenten magnetischen Dipolmomente angibt, ändert sich der Zusammenhand zu
            \begin{equation}
                \vec{B} = \mu_0 (\vec{H} + \vec{M})
            \end{equation}
            Zusätzlich lässt sich nun die magnetische Suszeptibilität $\chi$ nutzen um beispielsweise die Relation zwischen $\vec{H}$ und $\vec{M}$ zu beschreiben
            \begin{equation}
                \vec{M} = \chi \vec{H}
            \end{equation}
            Die magnetische Suszeptibilität beschreibt dabei wie stark dich die Magnetisierung in einem
            Material bei angelegetem Magnetfeld ändert, bzw wie viele der Magnetischen momente sich ausrichten.
            Dabei Unterscheidet man im Allgemeinen zwischen
            \begin{itemize}
                \item $\chi > 0$ Paramagneten
                \item $\chi < 0$ Diamagneten
            \end{itemize}
            Daraus folgt über
            \begin{equation}
                \mu = \mu_0 (1 + \chi)
            \end{equation}
            die Magnetsiche permeabilität $\mu [\frac{N}{A^2}]$, die das Verhältniss zwischen Magnetischer Flussdichte $B$ und magnetischer Feldstärke $H$ beschreibt
            $$ \mu = \frac{B}{H}$$
            Zuletzt gilt es noch das magnetische (Dipol-)Moment $\vec{\mu}$ [$\frac{A}{m^2}$] zu beschreiben, dieses ist ein Maß für die Stärke
            und Richtung eines Magnetischen Dipols. Es lässt sich jedoch auch für Ströme definieren, beispielsweise gilt für eine
            Leiterschleife
            \begin{equation}
                \vec{\mu} = n I \vec{A}
            \end{equation}
            oder Allgemeiner 
            \begin{equation}
                \vec{D} = \vec{\mu} \times \vec{B}
            \end{equation}.
            Erweitert man das Konzept einer Leiterschleife zu einer Reihe derselbigen, erhält das Magnetfeld einer Spule mit Windungen N
            \begin{equation}
                B = N \frac{\mu_0 I}{l}
            \end{equation}
        