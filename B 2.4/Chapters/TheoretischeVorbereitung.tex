\section{Theoretische Vorbereitung}
    \begin{itemize}
        \item für die Sekundärspule ist keine homogene Bewickelung notwendig, warum?
        \item 
    \end{itemize}
    \subsection{Grundlegende Beziehungen}
        \subsubsection{Magnetfelder}
            In der Physik wird das Vektorielle Magnetfeld meist als B-Feld bezeichnet. Es besitzt den Formelbuchstaben
            $\vec{B}$ und wird genauer auch als Magnetische Flussdichte bezeichnet und es wird in der Regel in Tesla angegeben.
            Das B-Feld ist jedoch keine Messbare größe, die zugehörige Messgröße zum B-Feld ist die Magnetische Feldstärke $\vec{H}$.
            Die Magnetische Feldstärke $H$ wird in Ampere pro Meter angegeben ($\frac{A}{m}$) und ist im Vaccum auch Direkt proportional
            zum B-Feld. Dort gilt
            \begin{equation}
                \vec{B} = \mu_0 \vec{H}
            \end{equation}
            $\mu_0$($\eqsim 4\pi ×10−7 \frac{N}{A^2}$) bezeichnet dabei die Magnetische permeabilität von Vaccum.
            In Anwesenheit eines Material mit Magnetisierungs Vektor $M [\frac{A}{m}]$, welcher die dichte der im Material induzierten oder permanenten magnetischen Dipolmomente angibt, ändert sich der Zusammenhand zu
            \begin{equation}
                \vec{B} = \mu_0 (\vec{H} + \vec{M})
            \end{equation}
            Zusätzlich lässt sich nun die magnetische suszeptibilität $\chi$ nutzen um beispielsweise die Relation zwischen $\vec{H}$ und $\vec{M}$ zu beschreiben
            \begin{equation}
                \vec{M} = \chi \vec{H}
            \end{equation}
        