\section{Theoretische Vorbereitung}
    \begin{itemize}
        \item für die Sekundärspule ist keine homogene Bewickelung notwendig, warum?
        \item 
    \end{itemize}
    \subsection{Grundlegende Beziehungen}
        \subsubsection{Magnetfelder}
            In der Physik wird das Vektorielle Magnetfeld meist als B-Feld bezeichnet. Es besitzt den Formelbuchstaben
            $\vec{B}$ und wird genauer auch als Magnetische Flussdichte bezeichnet und es wird in der Regel in Tesla angegeben.
            Das B-Feld ist jedoch keine Messbare größe, die zugehörige Messgröße zum B-Feld ist die Magnetische Feldstärke $\vec{H}$.
            Die Magnetische Feldstärke $H$ wird in Ampere pro Meter angegeben ($\frac{A}{m}$) und ist im Vaccum auch Direkt proportional
            zum B-Feld. Dort gilt
            \begin{equation}
                \vec{B} = \mu_0 \vec{H}
            \end{equation}
            $\mu_0$($\eqsim 4\pi ×10^{-7 }\frac{N}{A^2}$) bezeichnet dabei die Magnetische permeabilität von Vaccum.
            In Anwesenheit eines Material mit Magnetisierungs Vektor $M [\frac{A}{m}]$, welcher die dichte der im Material induzierten oder permanenten magnetischen Dipolmomente angibt, ändert sich der Zusammenhand zu
            \begin{equation}
                \vec{B} = \mu_0 (\vec{H} + \vec{M})
            \end{equation}
            Zusätzlich lässt sich nun die magnetische Suszeptibilität $\chi$ nutzen um beispielsweise die Relation zwischen $\vec{H}$ und $\vec{M}$ zu beschreiben
            \begin{equation}
                \vec{M} = \chi \vec{H}
            \end{equation}
            Die magnetische Suszeptibilität beschreibt dabei wie stark dich die Magnetisierung in einem
            Material bei angelegetem Magnetfeld ändert, bzw wie viele der Magnetischen momente sich ausrichten.
            Dabei Unterscheidet man im Allgemeinen zwischen
            \begin{itemize}
                \item $\chi > 0$ Paramagneten
                \item $\chi < 0$ Diamagneten
            \end{itemize}
            Daraus folgt über
            \begin{equation}
                \mu = \mu_0 (1 + \chi)
            \end{equation}
            die Magnetsiche permeabilität $\mu [\frac{N}{A^2}]$, die das Verhältniss zwischen Magnetischer Flussdichte $B$ und magnetischer Feldstärke $H$ beschreibt
            $$ \mu = \frac{B}{H}$$
            Zuletzt gilt es noch das magnetische (Dipol-)Moment $\vec{\mu}$ [$\frac{A}{m^2}$] zu beschreiben, dieses ist ein Maß für die Stärke
            und Richtung eines Magnetischen Dipols. Es lässt sich jedoch auch für Ströme definieren, beispielsweise gilt für eine
            Leiterschleife
            \begin{equation}
                \vec{\mu} = n I \vec{A}
            \end{equation}
            oder Allgemeiner 
            \begin{equation}
                \vec{D} = \vec{\mu} \times \vec{B}
            \end{equation}.
            Erweitert man das Konzept einer Leiterschleife zu einer Reihe derselbigen, erhält das Magnetfeld einer Spule mit Windungen N
            \begin{equation}
                B = N \frac{\mu_0 I}{l}
            \end{equation}
    \subsection{Magnetismus ohne Ordnungsphänomene}
        \subsubsection*{gyromagnetisches Verhältniss}
            das gyromagnetische Verhältnis eines magnetischen Moments wird beschrieben über das Verhältnis
            des magn. Moments und dessen Drehimpuls
            \begin{equation}
                \gamma = \frac{\mu}{L}
            \end{equation}
        \subsubsection*{\textit{Spin-,Bahnmagnetismus und Landé-Faktor}}
            Alle Teilchen die sowohl einen Drehmoment als auch eine el. Ladung besitzten, haben ein magn. Dipolmoment.
            \begin{equation}
                \vec{\mu_l} = \frac{q}{2m} \vec{l}
            \end{equation}
            dies wird allg. als Bahnmagnetismus beschrieben. Analog dazu gilt eine ähnliche formel für den Spin
            \begin{equation}
                \vec{\mu_s} = \frac{q\cdot g_s}{2m} \vec{s}
            \end{equation}
            bei dem der anomale Spin-g-Faktor berücksichtigt wird.
            Das Bohr'sche Magneton ist der Betrag des magnetischen Moments, welches ein Elektron
            mit $l=1$ erzeugt, also der Grundzustand.
            \begin{equation}
                \mu_B = \frac{e \hbar}{2 m_e} = 5,7883818060(17)\cdot 10^{-5} \frac{eV}{T}
            \end{equation}
            Das magnetische Moment eines Elektrons wird meist durch ein vielfaches des Bohr'schen
            Magneton beschrieben.
            Der Landé-Faktor ist das Verhältnis der Diskrepanz, zwischen dem gemessenen magnetischen Moments
            und dem errechneten (kl. Physik), eines Elektrons. $g_s \approx 2$ für ein Elektron.
        \subsubsection*{Diamagnetismus}
            Bei einem Diamagneten richten sich die inneren magnetischen Momente entgegengesetzt
            zu einem äußeren Magnetfeld aus. Dies passiert auf Grund der unten erklärten Lenz'schen Regel.
            Außerhalb eines magnetischen Feldes sind Diamagneten
            nicht magnetisch. Der einzige ideale Diamagnet ist ein Supraleiter, bei denen gilt $\chi = -1$           
        \subsubsection*{Lenz'sche Regel}    
            Die Lenz'sche Regel besagt, dass eine Änderung eines Magnetfeldes $\vec{B}$ einem der Änderung
            entgegenwirkenden Strom induziert
        \subsubsection*{Langevin-Gleichung}
            \hl{fehlt alles}
            Stochaistische Differentialgleichung oder so 
        \subsubsection*{Paramagnetismus}
            Ein Paramagnet besitzt eine positive Suszeptibilität, ist jedoch ohne äußeres 
            Magnetfeld nicht magnetisch. In einem Paramagneten richten sich die inneren magnetischen 
            in Richtung des Feldes aus.
            \hl{Beitraege}
            Der Pauli-Paramagnetismus entsteht durch freie Elektronen in Metallen, die über ihren Spin 
            in magnetisches Moment besitzen. Da jedoch, aufgrund des Pauli-Prinzips, nur angeregte
            Leitungselektronen über der Fermi Energie sich nach dem Magnetfeld ausrichten können, 
            ist die Anzahl der beitragenden Elektronen proportional zur Materialabhängigen Fermi-Temperatur.
            \begin{equation}
                \chi_{Pauli} \thicksim \frac{C}{T} \cdot \frac{T}{T_{F}} = \frac{C}{T_F}\thicksim 10^{-6 ... -5}
            \end{equation}
            mit C der Curie-Konstante, T der Temperatur und $T_F$ der Fermi-Temperatur.
            
