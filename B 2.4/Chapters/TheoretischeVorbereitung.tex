\section{Theoretische Vorbereitung}
    \subsection{Grundlegende Beziehungen}
        \subsubsection{Magnetfelder}
            Die magnetische Flussdichte $\vec{B}$ ist ein Vektorfeld, das im SI-System die Einheit Tesla hat.
            Das $\vec{B}$ - Feld ist jedoch keine direkt messbare Größe, die zugehörige Messgröse ist die magnetische Feldstärke $\vec{H}$.
            Diese wird in Ampere pro Meter angegeben ($\frac{A}{m}$) und ist im Vakuum direkt proportional
            zum $\vec{B}$ - Feld. Dort gilt:
            \begin{equation}
                \vec{B} = \mu_0 \vec{H}
            \end{equation}
            $\mu_0 = 4\pi \cdot 10^{-7}\frac{N}{A^2}$ bezeichnet dabei die magnetische Permeabilität des Vakuums.
            In Anwesenheit eines Materials mit der Magnetisierung $\vec{M} [\frac{A}{m}]$, welche die räumliche Dichte der im Material induzierten oder permanenten magnetischen Dipolmomente $\vec{\mu}$ angibt ($\vec{M} = \frac{d\vec{\mu}}{dV}$), ändert sich der Zusammenhang zu
            \begin{equation}
                \vec{B} = \mu_0 (\vec{H} + \vec{M})
            \end{equation}
            Zusätzlich lässt sich die magnetische Suszeptibilität $\chi$ nutzen, um beispielsweise die Relation zwischen $\vec{H}$ und $\vec{M}$ zu beschreiben:
            \begin{equation}
                \vec{M} = \chi \vec{H}
            \end{equation}
            Die magnetische Suszeptibilität beschreibt dabei, wie stark sich die Magnetisierung in einem
            Material, das sich in einem externen Magnetfeld befindet, ändert, bzw. wie viele der magnetischen Momente sich entlang des externen Feldes ausrichten.
            Dabei Unterscheidet man im Allgemeinen zwischen
            \begin{itemize}
                \item $\chi > 0$ Paramagneten
                \item $\chi < 0$ Diamagneten
            \end{itemize}
            Daraus ergibt sich über
            \begin{equation}
                \mu = \mu_0 (1 + \chi)
            \end{equation}
            die magnetische Permeabilität $\mu [\frac{N}{A^2}]$, die das Verhältnis zwischen magnetischer Flussdichte $B$ und magnetischer Feldstärke $H$ beschreibt:
            $$ \mu = \frac{B}{H}$$
            Das magnetische (Dipol-)Moment $\vec{\mu}$ [$\frac{A}{m^2}$] ist ein Maß für die Stärke
            und Richtung eines magnetischen Dipols. Es lässt sich jedoch auch für Ströme definieren, beispielsweise gilt für eine ebene
            Leiterschleife (Strom $I$) mit Fläche $A$ und Flächennormalenvektor $\vec{n_A}$:
            \begin{equation}
                \vec{\mu} = A I \vec{n_A}
            \end{equation}
            In einem externen $\vec{B}$ - Feld wirkt auf einen magnetischen Dipol $\vec{\mu}$ ein Drehmoment $\vec{D}$:
            \begin{equation}
                \vec{D} = \vec{\mu} \times \vec{B}
            \end{equation}.
            Erweitert man das Konzept einer Leiterschleife zu einer Reihe von Leiterschleifen, erhält man das Magnetfeld einer Spule der Länge $l$ mit $N$ Windungen, durch die ein Strom $I$ fließt:
            \begin{equation}
                B = N \frac{\mu_0 I}{l}
            \end{equation}
    \subsection{Magnetismus ohne Ordnungsphänomene}
        \subsubsection*{gyromagnetisches Verhältnis}
            Das gyromagnetische Verhältnis $\gamma$ eines magnetischen Moments $\mu$ wird beschrieben über das Verhältnis
            des magn. Moments zu dessen Drehimpuls $L$:
            \begin{equation}
                \gamma = \frac{\mu}{L}
            \end{equation}
        \subsubsection*{Spin-,Bahnmagnetismus und Land\'e-Faktor}
            Alle Teilchen, die sowohl einen Drehmoment $\vec{\ell}$ als auch eine elektrische Ladung $q$ besitzten, haben ein magnetisches Dipolmoment $\vec{\mu_{\ell}}$:
            \begin{equation}
                \vec{\mu_{\ell}} = \frac{q}{2m} \vec{\ell}
            \end{equation}
            Dies wird allgemein als Bahnmagnetismus beschrieben. Analog dazu gilt eine ähnliche formel für den Spin $\vec{s}$:
            \begin{equation}
                \vec{\mu_s} = \frac{q\cdot g_s}{2m} \vec{s}
            \end{equation},
            bei dem der anomale Spin-$g$-Faktor zu berücksichtigen ist.
            Das Bohr'sche Magneton $\mu_B$ ist der Betrag des magnetischen Moments, welches ein Elektron
            mit $\ell=1$ erzeugt, also im Grundzustand:
            \begin{equation}
                \mu_B = \frac{e \hbar}{2 m_e} = 5,7883818060(17)\cdot 10^{-5} \frac{eV}{T}
            \end{equation}
            Das magnetische Moment eines Elektrons wird meist durch ein Vielfaches des Bohr'schen
            Magnetons beschrieben.
            Der Land\'e-Faktor ist das Verhältnis der Diskrepanz zwischen gemessenem magnetischen Moment
            und dem \dq klassisch\dq errechneten eines Elektrons. Für Elektronen ist $g_s \approx 2$.
        \subsubsection*{Diamagnetismus}
            Bei einem Diamagneten richten sich die inneren magnetischen Momente entgegengesetzt
            zu einem äußeren Magnetfeld aus. Dies passiert aufgrund der unten erklärten Lenz'schen Regel.
            Außerhalb eines magnetischen Feldes sind Diamagneten
            nicht magnetisiert. Der einzige ideale Diamagnet ist ein Supraleiter, bei denen gilt $\chi = -1$
        \subsubsection*{Lenz'sche Regel}
            Die Lenz'sche Regel besagt, dass eine Änderung eines Magnetfeldes $\vec{B}$ einen der Änderung
            entgegenwirkenden Strom induziert, vgl. Abb. \ref{figLenz}.
            
            \begin{figure}[H]
                \centering
                \includegraphics{Images/lenzRegel.png}
                \label{figLenz}
                % https://de.wikipedia.org/wiki/Datei:Electromagnetic_induction.svg
            \end{figure}
        \subsubsection*{Langevin-Gleichung}
        Die Larmor-Frequenz $\omega_L$ des Elektrons ist gegeben über $\omega_L = |\gamma | B$, mit $\gamma = -\frac{e}{2m_e}$, woraus sich $\omega_L = \frac{eB}{2m_e}$ ergibt. Daraus ergibt sich für $Z$ Elektronen ein durch die Larmor-Präzession hervorgerufener Strom $I$:
        \begin{equation}
        	I = Q \cdot f
        \end{equation}
        Dabei ist $Q = -Ze$ und die Frequenz $f = \frac{\omega_L}{2\pi}$:
        \begin{equation}
        	I = \frac{Ze^2 B}{4\pi m_e}
        \end{equation}
Daraus resultiert wiederum ein magnetisches Moment $\mu = I \cdot A$ (vgl. ebene Leiterschleife), wobei für die Fläche $A = \pi \langle \rho^2 \rangle$, mit dem mittleren quadratischen Abstand der Elektronen vom Kern $\langle \rho^2 \rangle$, angesetzt werden kann.\\
Nutzt man nun noch aus, dass $\langle \rho^2 \rangle = \langle x^2 \rangle + \langle y^2 \rangle$ und $\langle r^2 \rangle = \langle x^2 \rangle + \langle y^2 \rangle + \langle z2 \rangle$, so kann $\langle \rho^2 \rangle$ ersetzt werden durch $\frac{2}{3} \langle r^2 \rangle$. Für $N$ Atome pro Einheitsvolumen ergibt sich mit diesen überlegungen für deren magnetische Suszeptibilität $\chi = \frac{N\mu}{B}$ die Langevin - Gleichung:
	\begin{equation}
		\chi = - \frac{NZe^2}{6m_e} \langle r^2 \rangle
	\end{equation}
        \subsubsection*{Paramagnetismus}
            Ein Paramagnet besitzt eine positive Suszeptibilität, ist jedoch ohne äußeres
            Magnetfeld nicht magnetisiert. In einem Paramagneten richten sich die inneren magnetischen Momente
            in Feldrichtung aus.
            \subsubsection*{Einzelbeiträge}
            Der Pauli-Paramagnetismus entsteht durch freie Elektronen in Metallen, die über ihren Spin
            ein magnetisches Moment besitzen. Da jedoch, aufgrund des Pauli-Prinzips, nur angeregte
            Leitungselektronen mit Energie oberhalb der Fermi-Energie sich nach dem Magnetfeld ausrichten können,
            ist die Anzahl der beitragenden Elektronen proportional zur Materialabhängigen Fermi-Temperatur:
            \begin{equation}
                \chi_{Pauli} \thicksim \frac{C}{T} \cdot \frac{T}{T_{F}} = \frac{C}{T_F}\thicksim 10^{-6 ... -5}
            \end{equation}
            mit der Curie-Konstante $C$, der Temperatur $T$ und der Fermi-Temperatur $T_F$.
            %TO DO: weitere Beitraege


    \subsection{Magnetismus mit Ordnung}
        \subsubsection*{Wechselwirkung}
            Die Magnetische Ordnung kommt zustande über die sogennante Austauschwechselwirkung, bei welcher sich
            die atomaren Gesamtdrehmomente ein energietisches Minimum erstrebend zueinander Ausrichten. Dabei ist
            das Pauli-Prinzip ein entscheidener Faktor, da durch dieses bei Fermionen viele der übergänge verboten sind.
            Ein häufig verwendetes Modell zur Beschreibung dieser Austauschwechselwirkung ist das Ising-Modell.
        \subsubsection*{Anisotropie}
            Magnetisch leichte Achsen bzw Schwere Achsen beschreiben Achsen entlang denen die Magnetisierung bevorzugt
            bzw. nur unter großem Aufwand stattfindet. Die Anisotropieenergie bezeichnet dabei die notwendige Energie
            um eine Magnetisierung von einer leichten in eine schwere Achse zu drehen.\\
            Anisotropieenergie spaltet sich dabei grob in 3 Terme auf:
            \begin{itemize}
                \item \textbf{Magnetokristalline Anisotropie}, hervorgerufen durch die Anisotropie einer Kristallstruktur
                \item \textbf{Formanisotropie}, resultierend aus der Probengeometrie
                \item \textbf{induzierte Anisotropie}, durch Spannungen aufgrund von Unregelmäßigkeiten.
            \end{itemize}
        \subsubsection*{Ferro-, Ferri-, Antiferromagnetismus}
            Ferromagneten Zeichnen sich darüber aus, dass sich sog. Weiss'sche Bezirke bilden. In diesen Bezirken
            kommt es zu einer spontanten Magnetisierung der magn. Dipole. Das bedeutet, dass sich die Dipole entlang einer
            gemeinsamen Vorzugsrichtung ausrichten. Dieses Verhalten kann spontan und unabhängig von einem äusseren Magnetfeld
            auftreten, wird es jedoch durch ein solches induziert bleibt es im Anschluss auch in Abwesehnheit eines solchen bestehen.
            Beim Ferrimagnetismuss verhält es sich recht Analog, jedoch sind einige Dipole entgegengesetzt ausgerichtet.
            Dabei kommt es jedoch nicht zu einer Aufhebung der Gesamtmagnetisierung, da in ferrimagnetischen Materialen
            eine Ausrichtung bevorzugt, d.h. stärker ist, wodurch eine Gesamtmagnetisierung entsteht.
            Bei antiferromagnetischen Materialen verhält es sich wie bei Ferrimagnetischen Materialen ohne Vorzugsrichtung,
            wodurch sich insgesamt keine Magnetisierung durchsetzt, es bleibt unmagnetisch.
            \begin{figure}[H]
                \centering
                \includegraphics{Images/ferroferrianti.png}
            \end{figure}
        \subsubsection*{Ferromagnetismus}
            Wie bereits angesprochen kommt es in ferromagnetischen Materialien zu sog. Weiss'schen Bezirken,
            oder auch Domänen, die sich aufgrund der Kristallstruktur und dessen inhomogenitäten ausbilden. Dabei
            hängt die Vorzugsrichtung des jeweiligen Bezirks vom Kristallgitter der Probe ab. Solche
            Bezirke erstrecken sich von etwa 10 bis 1000$\mu m$ in linearer Ausdehnung.\\
            An den übergängen von zwei Weiss'schen Bezirken bilden sich Bloch-Wände, in denen die magnetischen Momente stückweise von den Ausrichtungen innerhalb der angrenzenden Weiss-Bezirke ineinander übergehen.\\
            In Anwesenheit eines externen Feldes verschieben sich diese Bloch-Wände nun zugunsten einer Ausrichtung der magnetischen Momente entlang des externen Feldes. Wird die externe Feldstärke erhöht, kann es dabei zu Barkhausensprüngen kommen, bei welchen sich komplette Weissbezirke spontan ummagnetisieren. Diese Ummagnetisierungen sind stets reversibel, indem man die Richtung des externen Magnetfelds umkehrt. Dabei treten Hysterese-Effekte auf, die wir im Folgenden beschreiben.
           % \hl{todo: Energie, Wandversch., Drehung, Barkhausenspruengen, Reversibilitaet}
        \subsubsection*{Hysteresekurve}
            Hysteresekurven treten in vielen Bereichen der Physik auf, in diesem Fall wird jedoch die Magnetisierung
            der Probe in Abhängigkeit eines äusseren Feldes aufgetragen.\\
            Führt man eine Solche Messung durch, beginnt der Graph mit der sogennanten Neukurve, dabei richten sich
            die Weiss'schen Bezirke das erste mal aus bis zum Punkt der Sättigung. Misst man dann die Magnetisierung bei einem Abnehmenden Magnetfeld
            kommt man zunächst zum Remanenzfeld, welches bestehenbleibt obwohl kein äusseres Feld mehr angelegt ist. Wird die externe Feldstärke nun entgegen der anfänglichen Ausrichtung erhöht, nimmt die Restmagnetisierung weiter ab, bis
            sie beim Erreichen des Koerzitivfeldes schließlich verschwindet und es erneut zur spontanen Magnetisierung kommt und der Vorgang von neuem Beginnt.
            \begin{figure}[H]
                \centering
                \includegraphics[width=0.9\textwidth]{Images/hyster.png}
                %https://www.google.com/url?sa=i&url=https%3A%2F%2Fwww.elektroniktutor.de%2Felektrophysik%2Fmagkurve.html&psig=AOvVaw202TVqiaXyCzzC35HNuQiG&ust=1622981447740000&source=images&cd=vfe&ved=0CAMQjB1qFwoTCLi505e7gPECFQAAAAAdAAAAABAD
            \end{figure}

            Zeigen Sie, dass der Flächeninhalt der Hysteresekurve ein Maß fur die Energie ist, die beim einmaligen Umfahren der Kurve als Wärme
            auftritt (Verlust).Je höher die Energie, die aufgebracht werden muss um die Weißschen Bezirke auszurichten, desto größer ist die
            Remanenz und Koerzitivfeldstärke. Und je größer Remanenzund Koerzitivfeldstärke, desto größer die Fläche unter der Hysteresekurve.
            Daraus folgt, der Flächeninhalt ist proportional zur benötigten Energie, die beim einmaligen Umfahren der Kurve benötigt wird
        \subsubsection*{magn. Hart und Weich}
            Magnetisch weiche Stoffe zeichen sich durch eine besonders leichte Magnetisierbarkeit aus, das bedeutet,
            dass sie eine kleineres Koerzitivfeld benötigen um ihre Magnetisierung zu ändern. Anders ist es bei Harten
            magnetischen Stoffen bei denen die Magnetisierung besonders Schwer ist.
            \begin{figure}[H]
                \centering
                \includegraphics[width=0.9\textwidth]{Images/übersicht_Koerzitivfeldstärke.png}
                %https://de.wikipedia.org/wiki/Magnetwerkstoffe#/media/Datei:%C3%9Cbersicht_Koerzitivfeldst%C3%A4rke.svg
            \end{figure}
        \subsubsection*{Ferrite}

        \subsubsection*{Temperatureinfluss}
            Bei steigenden Temperaturen führt die thermische Energie zu bewegung im Material, wodurch die Ordnung der
            Dipole gestört wird. Je höher die Temperatur desto schwerer wird es für die Dipole ihre Ausrichtung entlang
            der Feldlinien bei zubehalten. Steigt die Temperatur über die sog. Curie Temperatur verliert das Material
            die Eigenschaft der Magn. Ausrichtung und wird nichtmagnetisch
        \subsubsection*{Phasenübergang}
            %https://sci-hub.se/https://link.springer.com/chapter/10.1007/978-3-642-82138-7_1
            Bei Thermondynamischen Phasenübergänge spricht man häufig über eine Spontane änderung der Freien Energie F.
            Bei ihnen wird zwischen übergängen erster und zweiter Ordnung Unterschieden. übergänge erster Ordnung
            arbeiten mit latenter Wärme, also ohne Temperaturänderung. Dabei tauscht das System eine Feste menge an energy
            mit der Umgebung aus.\\
            Bei übergängen 2ter ordnung ist keine solche spontanität zu erkennen, sie sind auch sog. Kontinuierliche Phasenübergang
            \begin{figure}[H]
                %https://en.wikipedia.org/wiki/Phase_transition
                \centering
                \includegraphics[width=0.9\textwidth]{Images/waterphase.jpg}
            \end{figure}
    \subsection{Entmagnetisierungsfaktor}

        \subsubsection*{Entmagnetisierung}
            Eine Entmagnetisierung ist ein Vorgang, indem das Magnetfeld eines Magneten verschwindet.
            Die Ursache hierfür können: Erschütterung, Hitze sein. Aber meist wird ein Material
            durch ein zuerst starkes Wechsel-Magnetfeld, welches aber mit der Zeit abklingt, verwendet.
            \begin{figure}[H]
                \centering
                \includegraphics{images/Ringkern}
            \end{figure}
        \subsubsection*{Herleitung Entmagnetisierungsfaktor}
            Man nimmt an, dass die Lücke des Rings sehr klein ist und die Feldlinien sich homogen
            fortsetzen. Dann vergleicht man die Erregerfelder mit und ohne Luftspalt. Dies geht, da
            bei beiden die gleiche Stromstärke verwendet wird.
            \begin{equation}
                H \cdot L = H_R \cdot L_R + H_L \cdot L_L
            \end{equation}
            mit $L = 2 \pi r$ der Umfang des Kerns. $L_L, L_R$ sind die Länge der Lücke bzw. des restlichen Kerns.
            $H_R, H_L$ sind die Erregerfelder im Ring und in der Lücke.
            Mit den Maxwellgleichungen und dem Stoke'schen Satz folgt:
            \begin{equation}
                B_R \cdot F = B_L \cdot F
            \end{equation}
            mit F ist die Querschnittsfläche des Rings. Es gilt allg.
            \begin{equation}
                B = \mu_0 (H + M)
            \end{equation}
            da Luft als Vakuum genähert werden kann, ist $M=0$ in der Lücke.
            Durch diese Gleichungen ergeben sich für $H_R$ bzw. $H_L$:
            \begin{align*}
                H_R = H - \frac{L_L}{L} M\\
                H_L = H + \frac{L_L}{L} M
            \end{align*}
            das entmagnetisierende Feld ist damit:
            \begin{equation}
                H_e = \frac{L_L}{L} M
            \end{equation}
            mit dem Entmagnetisierungsfaktor $N = \frac{L_L}{L}$
        \subsubsection*{gescherte Hysteresenkurven}
            %https://www.google.com/url?sa=i&url=https%3A%2F%2Fsilo.tips%2Fdownload%2Fmagnetisierungskurven-eines-ferrits-1&psig=AOvVaw1-iHGdJClryjKD8ByJthm3&ust=1623062593991000&source=images&cd=vfe&ved=0CAIQjRxqFwoTCMjA4MHpgvECFQAAAAAdAAAAABAO
            Da bei steigender Spaltbreite das entmagnetisierende Feld $H_{ent}$ immer größer wird, werden wir das erzeugende Feld $H_{ohne}$
            auch steigern müssen. Wenn man in einem M-H-Diagramm eine Hysterese für einen Ringkern
            ohne Luftspalt aufnimmt, so ist $H_{E}=H_{ohne}$. Für Hyteresen mit Ringkernen,
            die einen Luftspalt besitzen, steigen die $H_{ohne}$-Werte, die die gleiche Magnetisierung
            bzw. das gleiche $_{E}$ erzeugen, immer weiter an, sodass die Hysterese nach rechts
            geschert wird. Man nennt diese Hysteresen daher auch gescherte Hysteresen.
            \begin{figure}
                \centering
                \includegraphics[width=0.9\textwidth]{Images/geschert.png}
            \end{figure}

        \subsubsection*{Einfluss der Entmagnetisierung auf $\chi$}
            Die sog. scheinbare Suszeptibilität lässt sich definieren über
            \begin{equation}
                \chi_s = \frac{M}{H_{ohne}}
            \end{equation}
            die wahre Suszeptibilität hingegen lässt sich über das Effektive feld definieren als
            \begin{equation}
                \chi_w = \frac{M}{H_{E}}
            \end{equation}
            Nutzt man nun die Beziehung zwischen dem Effektiven Feld und dem Feld ohne Material
            \begin{equation}
                H_E = H_{ohne} - N\cdot M
            \end{equation}
            folgt damit
            \begin{equation}
                \frac{M}{\chi_w} = \frac{M}{\chi_s} - N \cdot M
            \end{equation}
            \begin{equation}
                \chi_s = \frac{\chi_w}{1+N\cdot \chi_w}
            \end{equation}

