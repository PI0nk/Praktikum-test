\section{Einleitung}
    Im folgenden Versuch soll näher auf die Magnetisierung eines Ferrit Ringkerns und die generellen ferromagnetischen Ordnungsphänomene
    eingegangen werden. Um diese Phänomene zu beobachten nutzt man in diesem Versuch zwei verschiedene Aufbauten, die je aus einer Primärspule und einer Sekundärspule bestehen,
    um die Magnetisierung des Ringkerns messbar zu machen. Um die ferromagnetischen Ordnungsphänomene zu betrachten werden für die Verschiedenen aufbauten je die Hystereseschleifen aufgenommen
    um Informationen über Remanenz, Koerzitivfeldstärke und Sättigungsmagnetisierung zu erhalten. 