\section{Durchführung}
    Der Versuch besteht aus zwei Versuchsteilen.
    \subsection{Messung am beheizbaren Ringkern}
        In dem ersten Versuchsteil wird der beheizbare Ringkern angeschlossen. 
        Um die Kenngrößen(Remanenzfeldstärke, Koerzitivfeldstärke, maximale Magnetisierung) des Kerns zu bestimmen,
        werden Magnetisierungskurven bei verschiedenen maximal Stromstärken I$_max$ aufgenommen:
        3A, 1A, 300mA, 100mA. Um eine gut aufgelöste Magnetisierungskurve auf zu zeichnen, 
        erfolgen die Messungen im freilaufenden Betrieb bei langsamer Geschwindigkeit.
        \\
        Darauf folgend wird die Kommutierungskurve aufgenommen, dies muss im Phase-locked-Betrieb
        geschehen. Die Phase wird dabei so eingestellt, dass man sich am oberen Umkehrpunkt der Hysteresekurve
        befindet. Man nimmt zwei Kommutierungskurven auf, einmal wird der Strom von I$_max$ = 100mA langsam auf 0A runter geregelt und
        bei der zweiten Kurve von I$_max$ = 3A auf 0A.
        In dem letzten Versuchsteil am beheizbaren Ringkern wird der Temperaturverlauf der
        maximalen Magnetisierung gemessen. Die Phase wird wieder auf die Spitze der Hysteresekurve
        eingestellt, bei I$_max$ = 3A.

    \subsection{Messung am Ringkern mit Spalt}
        Als erstes wird eine Magnetisierungskurve bei 0.94A aufgenommen, um wie im ersten Versuchsteil
        die Kenngrößen zu bestimmen. \\
        Um den Entmagnetisierungsfaktor zu bestimmen, werden Magnetisierungskurven bei unterschiedlichen
        Spaltbreiten gemessen. Dafür werden Plättchen mit bekannter Dicke benutzt, um einen exakten Spalt zu erzeugen.
        Es wird mit der größten Spaltbreite begonnen und der Strom auf den Größtmöglichen ($\leq 3A$) eingestellt.
        Mit Hilfe des Oszilloskops kann sicher gestellt werden, dass sich die maximale Magnetisierung der Messungen
        nicht verändert.