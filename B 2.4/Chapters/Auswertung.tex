\section{Auswertung}
    \subsection{Kenngrößen des Beheizbaren Ringkern}
        Zunächst wurde für die Beheizbare Spule mit Ringkern bei Raumtemperatur für verschiedenen Stromstärken je eine Hysteresenkurve
        aufgenommen um daraus anschließend die Remanenz-, Koerzitivfeldstärke und Maximale Magnetisierung zu bestimmen.\\
        Um die aufgenommen Spannungen der Gleichrichter verwenden zu können mussten diese zunächst in die Entsprechenden Magnetfeldstärken sowie Magnetisierungen umgerechnet werden.
        Für das Magnetfeld einer Spule gilt
        \begin{equation}
            H = \frac{n}{2\pi r}\frac{I_m}{U_m}U
        \end{equation}
        und nach Formel 2.5 aus der Anleitung für die Magnetisierung
        \begin{equation}
            M = \frac{U}{47\cdot 4\cdot \nu n_s q \mu_0}
        \end{equation}
        Mit $\nu = 50Hz$.\\
        Anschließend konnten die Kenngrößen aus folgenden Graphen abgelesen werden
        \begin{figure}[H]
            \centering
            \includegraphics[width=\textwidth]{Images/Teil1.PNG}
            \label{Hysteresen}
            \caption{Hysteresenkurven bei Stromstärken 100mA,300mA,1A und 3A}
        \end{figure}
        Daraus lies sich folgende Tabelle zusammenstellen
        \begin{table}[H]
            \centering
            \begin{tabular}[]{c|c|c|c}
                Stromstärke [A] & Remanenzfeldstärke [A/m] & Koerzitivfeldstärke [A/m] & maximale Magnetisierung [A/m] \\
                \hline
                0.1 & 8.36e4 & 8.3 & 17.12e4 \\
                0.3 & 10.66e4 & 11.33 & 24.26e4 \\
                1    & 9.5e4 & 10.99 & 28.84e4 \\
                3    & 7.15e4 & 6.86 & 30.86e4 \\
            \end{tabular}
            \caption{Kenngrößen des beheizbaren Ringkern bei Raumtemperatur}
        \end{table}
        Im allgemeinen entspricht die Tabelle unseren qualitativen Erwartungen, die maximale Magnetisierung steigt mit steigender Stromstärke und sowohl Remanenzfeldstärke als auch Koerzitivfeldstärke bleiben in etwa konstant. Die Fluktuation von sowohl Remanenzfeldstärke als auch Koerzitivfeldstärke
        wirkt hier jedoch fehl am platz, da diese nicht von der Stromstärke abhängen, was eine Messungenauigkeit impliziert.
    \subsection{Kommutierungskurve und Suszeptibilität}
        Um die Suszeptibilität des Systems zu bestimmen wurde durch die Variation des Primärspulenstroms jeweils eine Kommutierungskurve für 1 bis 0 A und 3 bis 0 A aufgenommen.
        Durch fitten des Models 
        \begin{equation}
            M(H) = A \cdot exp(-\frac{b}{H} + c) + d
        \end{equation}
        wurden zwei differenzierbare Äquivalente der aufgenommen Messdaten erzeugt um anschließend die Suszeptibilität in Abhängigkeit des angelegten Magnetfeldes zu plotten.
        \begin{figure}[H]
            \centering
            \includegraphics[width=\textwidth]{Images/Teil2Teil 2 1-0A.png}
            \label{Teil2-1A}
            \caption{Fit der Kommutierungskurve für 0.1 bis 0A}
        \end{figure}
        \begin{figure}[H]
            \centering
            \includegraphics[width=\textwidth]{Images/Teil2Teil 2 3-0A.png}
            \label{Teil2-3A}
            \caption{Fit der Kommutierungskurve für 3 bis 0A}
        \end{figure}
        \begin{figure}[H]
            \centering
            \includegraphics[width=\textwidth]{Images/Teil2Differentielle Suszeptibilität.png}
            \label{DiffSus}
            \caption{Plot der Suszeptibilität für beide Kommutierungskurven}
        \end{figure}
            Der Verlauf der differentiellen Suszeptibilität ist für beide Kurven physikalisch sinnvoll, da für beide Kurven ein Verlauf proportional zu $\frac{1}{H}$ zu erkennen ist ($\chi \propto \frac{1}{H}$). Auch ist zu erkennen,
            das für geringere Stromstärken die maximale Suszeptibilität abnimmt.
    \subsection{Temperaturabhängigkeit und Curie Temperatur}
        \begin{figure}[H]
            \centering
            \includegraphics[width=\textwidth]{Images/Teil3.png}
            \label{Temp}
            \caption{Temperaturabhängigkeit der Magnetisierung}
        \end{figure}
        Bei Aufnahme der Temperaturabhängigkeit der Magnetisierung ist deutlich zu Erkennen, das bei Steigenden Temperaturen durch die zunehmende innere Energie
        auch die Unordnung der internen Dipole zunimmt und somit die Magnetisierung abnimmt. Bei ca. 55°C verfällt die Magnetisierung vollständig. Dort ist die Curie-Temperatur erreicht.
        $$T_C  = (55 \pm 2)\text{°C}$$
        Dieser Wert ist bei uns leider starken Schwankungen unterlegen, da die Temperaturmessung sich teilweise sprunghaft änderte. Durch Auswertung der Rohdaten sind wir auf einen Wert von 55°C gestoßen, bei dem die Magnetisierung erstmals
        0 $\frac{A}{m}$ erreichte. Alternativ zur Auswertung durch die Rohdaten wurden hier auch das die Funktion
        $$ M(T) = M_0 * (1 - T/T_c)^b)$$
        gefittet, wodurch eine kritische Temperatur von 55.23 resultiert.
    \subsection{Ringkern mit Spalt}
        Für den Ringkern mit Spalt wurde zunächst analog zum ersten Teil der Auswertung eine Hysteresenkurve aufgenommen als Referenz für die folgenden Messungen mit Variablen Spaltbreiten.
        Diese Referenzmessung wurde für I=0.94A durchgeführt.\\
        I = 0.94A ergibt sich aus der Relation $\frac{17}{54} * 3A$, also dem Verhältnis der Windungen beider Spulen.
        Aus der Analyse des Graphen ergab sich
        \begin{figure}[H]
            \centering
            \includegraphics[width=\textwidth]{Images/Teil4.1.png}
            \label{Hyst4}
            \caption{Hysteresenkurve für den Ringkern mit Spalt bei 0.94A}
        \end{figure}
        Um nun anschließend den Entmagnetisierungsfaktor N als Funktion der Spaltbreite zu ermitteln wurden Hysteresenkurve bei verschiedenen Spaltbreiten und gleicher Maximalen Magnetisierung
        aufgenommen.
        \subsection{Entmagnetisierungsfaktor}
            \begin{figure}[H]
                \centering
                \includegraphics[width=0.8 \textwidth]{Images/Teil4.2.png}
                \caption{Hysteresekurven bei verschiedenen Spaltbreiten}
            \end{figure}
            In diesem Versuchsteil sollten nun Magnetisierungskurven bei gleichbleibender maximaler 
            Magnetisierung aufgenommen werden, damit aus den Scherungen der Kurven die Entmagnetisierungsfaktoren
            bestimmt werden können. Um die Entmagnetisierungsfaktoren berechnen zu können, mitteln wir zunächst 
            über die maximalen Magnetisierungen:
            \begin{align}
                \bar{M}_{max} = \sqrt{\frac{1}{6} \sum^6_{i=0} M_{max, i}}\\
                \Delta \bar{M}_{max} = \sqrt{\frac{1}{30} \sum^6_{i=0} (M_{max, i} - \bar{M}_{max})^2}
            \end{align}
            Damit erhalten wir für die mittlere maximale Magnetisierung: $\bar{M}_{max} = (129,808 \pm 788) \cdot 10^3 \frac{A}{m}$\\
            Um daraus die Entmagnetisierungsfaktoren zu bestimmen, wird das Entmagnetisierungsfeld $H_{ent} = H_a - H_i$ verwendet,
            mit $H_i$ dem inneren Feld (also bei Spaltbreite d = 0).\\
            Daraus werden die Entmagnetisierungsfaktoren wie folgt bestimmt:
            \begin{equation}
                N_e = \frac{H_a-H_i}{\bar{M}_{max}}
            \end{equation}
            Um die experimentellen Werte mit den theoretischen bestimmen zu können, müssen diese erst berechnet werden,
            was folgendermaßen geht:
            \begin{equation}
                N_t = \frac{L_L}{L} = \frac{d}{\pi R + d}
            \end{equation}
            \begin{table}[H]
                \centering
                \begin{tabular}[]{c|c|c}
                    d [mm] & $N_e$ & $N_t$\\
                    \hline
                    0.0     & 0.0       & 0.0   \\
                    0.075   & 0.0004    & 0.0016\\
                    0.125   & 0.0007    & 0.0026\\
                    0.2     & 0.0013    & 0.0042\\
                    0.5     & 0.0022    & 0.0105\\
                    1.0     & 0.0034    & 0.0208\\
                \end{tabular}
                \caption{Experimentelle und theoretische Werte für die Entmagnetisierungsfaktoren}
            \end{table}
            \begin{figure}[H]
                \centering
                \includegraphics[width=0.8 \textwidth]{Images/Teil4.2_Entmagnetisierungsfakoren.png}
                \caption{Entmagnetisierungsfaktoren aufgetragen gegen Spaltbreiten}
            \end{figure}
            Man erkennt aus Abbildung 9 und Tabelle 2, dass unsere experimentellen Werte deutlich zu klein sind und
            stark von den theoretisch berechneten Werten abweichen. Jedoch ist wie zu erwarten der Verlauf 
            nahezu linear, was darauf schließen lassen könnte, dass es ein System interner Fehler ist %\hl{?}
            .
            Die Abweichungen könnten auf fehlerhafte Annahmen zurückzuführen sein. Wir haben beispielsweise das Material im Spalt,
            als Vakuum genährt, wobei es sich jedoch um Kunststoff handelt. Außerdem haben wir angenommen, dass das Feld 
            im Inneren des Spaltes homogen weiterlaufen würde, was auch nicht der Fall ist.\\
            \\

            Alles in allem sind unsere Werte in diesem Versuchsteil deutlich zu klein, wobei jedoch
            der Verlauf wie zu erwarten linear ist. Auf Grund eines System internen Fehlers %\hl{?}
             sind die Messwerte
            um eine Größenordnung zu klein. 