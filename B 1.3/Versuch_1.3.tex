\documentclass{scrartcl}
\usepackage{amsmath}
\usepackage[ngerman]{babel}
\usepackage{amsfonts}
\usepackage{amssymb}
\usepackage{graphicx}
\usepackage{figsize}
\usepackage{float}
\usepackage{geometry}
\geometry{verbose,tmargin=2.5cm,bmargin=2.5cm,lmargin=2.5cm,rmargin=2.5cm}
\usepackage[format=plain,font=small,labelfont=bf]{caption}
\usepackage[OT2,T1]{fontenc}
\DeclareSymbolFont{cyrletters}{OT2}{wncyr}{m}{n}
\DeclareMathSymbol{\Sha}{\mathalpha}{cyrletters}{"58}
\selectlanguage{ngerman}
\begin{document}

\thispagestyle{empty}
\vspace*{\fill}
\begin{center}
	\Huge
	\textbf{Universität zu Köln}\\
	\LARGE
	\textbf{Institut für Astrophysik}\\
	\vspace{2cm}
	\textbf{Versuchsprotokoll}\\
	\vspace{0.5cm}
	\large
	\textbf{B1.3: Bestimmung der Elementarladung nach Millikan}\\
	\normalsize
	\vspace{2cm}
	\begin{tabular}{r l}
		Autoren: 	& Jesco Talies$^1$\\
					& Timon Danowski$^2$\\
		Durchgefuehrt am:	& 19.04.2021\\
		Betreuer:	& Marius Hermanns
	\end{tabular}
\end{center}
\vfill\footnotesize
$^1$ jtalies@smail.uni-koeln.de, Matrikel-Nr.: 7348338\\
$^2$ tdanowsk@smail.uni-koeln.de, Matrikel-Nr.: 7348629\\
\normalsize

\newpage 
\thispagestyle{empty}
\tableofcontents
\clearpage
\setcounter{page}{1}

\section{Versuchsvorbereitung}
	\subsection{Motivation}
		Der Millikan-Versuch wurde erstmals 1910 von Robert Millikan und Harvey Fletcher durchgeführt. Die Motivation des Versuchs war es, die Elementarladung 
		gut zu bestimmen. Aus dem Experiment kam außerdem hervor, dass es eine kleinste (elementare) Ladung gibt und das Ladungen
		quantiesiert sind. Man also jede vorhande Ladung als vielfaches der Elementarladung schreiben kann.
	\subsection{Die Elementarladung}
		Die Elementarladung ist die kleinst mögliche frei existierende Ladung. Ihr Wert beträgt:

			\begin{equation}
				e = 1,602 176 634 * 10^{-19} C
			\end{equation}
		Sie wird häufig verwendet um die Ladung eines einzelnen Elektrons bzw. Protons zu beschreiben.


		Die Ladung ist eine skalare, additive Größe (d.h. die Gesamtladung lässt sich als ein Vielfaches der Elementarladung beschreiben)
		Die Erkenntnis über die Quantelung der Ladung geht auf die Beobachtungen von Robert Millikan zurück.

	\subsection{Kräfte im Versuch}
		In diesem Versuch befinden sich Öltröpchen in einem Plattenkondensator unter dem Einfluss verschiedener Kräfte.
		\subsubsection{Coulomb Kraft}
			Die Coulomb Kraft beschreibt die Kraft auf ein geladenes Objekt in einem elektrischen Feld.
			In einem Plattenkondensator:
			\begin{equation}
				\vec{F}_C = Q \cdot \frac{U}{d} \vec{e}_z
			\end{equation}
			wobei Q die Ladung des Objektes ist, U die Spannung zwischen den Kondensatorplatten und d der Abstand der Kondensatorplatten ist.	
		\subsubsection{Gravitationskraft}
			Die Gravitationskraft beschreibt die Beschleunigung einer Masse in einem Gravitationsfeld eines Massereichen Objektes.
			In diesem Fall im Gravitationsfeld der Erde, gilt:
			\begin{equation}
				\vec{F}_G = mg \vec{e}_r
			\end{equation}
			wobei m die Masse des Objektes (hier das Öltröpchen) und g die lokale Erdbeschleunigung mit $g = 9,80665 \frac{m}{s^2}$ ist.
		\subsubsection{Auftriebskraft}
			Als Auftriebskraft wird eine der Gewichtskraft eines Körpers entgegenwirkende Kraft bezeichnet. Diese resultiert aus dem Druckunterschied zwischen Ober-und Unterseite des Körpers, welcher aus dem Schweredruckdes Mediums resultiert
			Ein Tröpchen mit dem Radius r erfährt in der Luft unter der Erdbeschleunigung eine Auftriebskraft:
			\begin{equation}
				\vec{F}_A = - \rho_{Luft} \frac{4}{3} \pi r^3 \vec{g}
			\end{equation}
			mit $\rho_{Luft}$ als Dichte der Luft
		\subsubsection{Reibungskraft}
			Reibungskräfte sind geschwindigkeitsabhängige Widerstandskräfte, die der Bewegung eines Körpers entgegenwirken.
			Man unterscheidet zwischen äußerer und innerer Reibung. Für diesen Versuch interessiert uns die innere Reibung, welche sich aus der Viskosität des Mediums ableitet.
			Die Stokes-Reibungskraft auf eine Kugel mit dem Radius r und Geschwindigkeit v, welche sich durch ein Medium der Viskosität $\eta$ bewegt ergibt sich zu
			\begin{equation}
				\vec{F}_R = -6 \pi r \eta \vec{v}
			\end{equation}
		\subsubsection{Cunningham-Korrektur}
			Wenn der Radius eines Teilchens im Bereich der mittleren freien Weglänge des umgebenden Mediums liegt, trifft das Stoke´sche Reibungsgesetz nicht mehr zu.
			Dies liegt daran, dass die Stoke´sche Reibung die äußere Reibung zwischen Öl und Luft vernachlässigt.
			Unter Verwendung der Cunningham-Korrektur ändert sich die Stoke´sche Reibung zu:
			\begin{equation}
				F_R = -6 \pi r \eta v \cdot (1 + A_c \frac{<l>}{r})^{-1}
			\end{equation}
			wobei $<l>$ die mittlere freie Weglänge und $A_c$ eine Konstante ist, für die wir den Wert $A_c = 1,26$ verwenden werden.
		\subsection{Herleitung der Formeln}
			Aus den wirkenden Kräften können wir uns jetzt Formeln für den Radius und die Ladung eines Öltröpchens herleiten.
			Im Kräftegleichgewicht ist die Summe aller wirkenden Kräfte gleich 0.
			Das Medium in dem sich die Tröpchen bewegen ist Luft, daher	$\rho_m  = \rho_{Luft}$ und $\rho_K = \rho_{Öl}$

			Ein Tröpchen steigt für
			\begin{equation}
				F_C + F_A > F_G
			\end{equation}
			wegen der entgegenwirkenden  Reibungskraft gleichmäßig mit:
			\begin{equation}
				v_{steig} = \frac{1}{6\pi \eta r} (Q \frac{U}{d}-\frac{4}{3}\pi r^3 g (\rho_{Öl}-\rho_{Luft}))
			\end{equation}
			Bei Umpolung der Kondensatorplatten ($F_C + F_A < F_G$), ergibt sich für die Sinkgeschwindigkeit:
			\begin{equation}
				v_{sink} = \frac{1}{6\pi \eta r} (Q \frac{U}{d}+\frac{4}{3}\pi r^3 g (\rho_{Öl}-\rho_{Luft}))
			\end{equation}

			Durch Subtraktion der beiden Gleichungen und umstellen nach r, erhält man:
			\begin{equation}
				r = \sqrt{\frac{9 \eta (v_{sink} - v_{steig})}{4 g (\rho_{Öl} - \rho_{Luft})}}
			\end{equation}

			Durch Addition der beiden Gleichungen, Einsetzen von 0 oben und anschließendem umstellen anch Q, erhält man:
			\begin{equation}
				Q = \frac{3 \pi d \eta r}{U} (v_{sink} + v_{steig})
				\rightarrow \frac{9}{2} \pi d \sqrt{\frac{\eta^3 (v_{sink} - v_{steig})}{g (\rho_{Öl} - \rho_{Luft})}} \frac{v_{sink} + v_{steig}}{U} 
			\end{equation}

			Mit der Cunningham Korrektur kommt man mit den selben Schritten auf den Radius:
			\begin{equation}
				r_C = \frac{A_c <l>}{2} + \sqrt{\frac{A_C <l>}{2}^2 +\frac{9 \eta (v_{sink} - v_{steig})}{4 g (\rho_{Öl} - \rho_{Luft})}}
					= \frac{A_c <l>}{2} + \sqrt{\frac{A_C <l>}{2} + r^2}
			\end{equation}
			und die Ladung:
			\begin{equation}
				Q_C = \frac{3 \pi \eta r_C d}{U} (1 + A_C \frac{<l>}{r_C})^{-1} (v_{sink} + v_{steig})
				\rightleftarrows Q \cdot \frac{r_C}{r} (1 + A_C \frac{<l>}{r_C})^{-1} 
			\end{equation}

\section{Durchführung}
	\subsection{Aufbau}
		\begin{figure}[H]
			\centering
			\includegraphics[width=1.0\textwidth]{Versuchsaufbau.PNG}
			\caption{Versuchsaufbau}
		\end{figure}
		\begin{itemize}
			\item (a) Mikroskop mit vorgeschalteter Kamera
			\item (b) Kammer mit Plattenkondensator
			\item (c) Ölstäuber mit Blasebalg
			\item (d) Kommutatorschaltung
		\end{itemize}
	\subsection{Versuchsdurchführung}
		Zu Beginn des Versuchs müssen sowohl Mikroskop als auch Kamera kalibiriert werden. Dafür wird zunächst die
		Spannungsversorgung der Mikroskopkamera angeschaltet und dessen Addresse und Konfiguration über die 'Commen Vision Blox Management Console'
		gesucht und gespeichert um im folgenden das Bild der Kamera im Programm 'MovieInteractive 2' zu betrachten
		und die Belichtungszeit und Verstärkung der Kamera zu kalibrieren. Hierbei sollte darauf geachtet werden, dass die 
		Belichtungszeit nicht zu lang ist, da dies zu "Verschmierung" der Tropfen im Bild führt. Auch sollte 
		die Verstärkung nicht zu groß gewählt werden um das Rauschen des Hintergrunds gering zu halten. Hat man 
		eine aktzeptable Konfiguration gewählt sollte die Brennebene des Mikroskops in die Mitte des Plattenkondensators
		verschoben werden um die Tröpfchen zu filmen.
		Um die Steig- und Sinkzeit eines Tröpfchens zu messen wird zunächst der Blasebald betätigt um einige
		Öltröpchen zu zerstäuben. Anschließend wird durch wiederholtes Umpolen des Kondensator ein geladenes Tröpfchen 
		identifiziert und die Aufnahme kann über den Computer gestartet werden. Bei der Aufnahme sollten mindestens fünf
		vollständige Steig- und Sinkvorgänge aufgezeichnet werden um die Fehler in der Auswertung gering zu halten.
		Hierzu wartet man bis das Tröpfchen eine der äusseren Begrenzungen der Milimeterskala überschreitet und polt dann
		das Feld des Kondensators um und lässt es über die gegenüberliegende Begrenzung wandern.
		Dieser Prozess wird für insgesamt 20 Tröpfchen wiederholt. Es ist dabei darauf zu achten stehts
		die Brennebene des Mikroskops nach zu justieren, da die Tröpfchen durch Verwirbellung und Stöße mit 
		Luftpartikeln leicht aus der Brennebene wandern und dann nicht länger auf der Aufnahme zu sehen sind.
\section{Auswertung}
	Die Fallzeiten der oben beschriebenen Messung können anschließend durch Frameweise betrachtung der Aufnahmen
	extrahiert werden.
	Aus unseren gemessenen Steig- und Sinkzeiten ergaben sich folgende Mittelwerte und deren Fehler
	\begin{equation}
		\bar{t} = \frac{1}{n} \cdot \sum_{i=1}^n t_i
	\end{equation}
	\begin{equation}
		\Delta \bar{t} = \sqrt{\frac{1}{n\cdot(n-1)}\cdot \sum_{i=1}^n{t_i - \bar{t}}}
	\end{equation}
	mit n = Anzahl der Messwerte (hier 5) und $t_i$ = Messwert.
	Wobei 
	\begin{equation}
		t = \frac{Frame(Ende)-Frame(Start)}{Framerate}
	\end{equation}
	mit Framerate = 32,792 Bilder/s, somit ist t in Sekunden.

	Die Geschwindigkeiten ergeben sich über
	\begin{equation}
		v = \frac{s}{t}
	\end{equation}
	mit $s = (0.96 \pm 0.01) \mu m$ und ein Fehler von
	\begin{equation}
		\Delta v = v \cdot \sqrt{(\frac{\Delta s}{s})^2 + (\frac{\Delta t}{t})^2} 
	\end{equation}

	Daraus lässt sich der Radius folgendermaßen berechnen
	\begin{equation}
		r = \sqrt{\frac{9\eta \cdot (v_{sink}-v_{steig})}{4 g \cdot (\rho_{Öl}-\rho_{Luft})}}
	\end{equation}
	mit $\eta = 1.81e-5 Pa\cdot s$, $g=9.81 \frac{m}{s^2}$, $\rho_{Öl} = 1030 \frac{kg}{m^3}$, $\rho_{Luft} = 1.29 \frac{kg}{m^3}$
	und den Fehler
	\begin{equation}
		\Delta r = \frac{r}{2} \cdot \sqrt{(\frac{\Delta \eta}{\eta})^2+\frac{\Delta v_{sink}^2+\Delta v_{steig}^2}{(v_{sink}-v_{steig})^2}}
	\end{equation}
	und ebenfalls die Ladung Q
	\begin{equation}
		Q = \frac{3\pi \cdot d\eta}{U} \cdot r \cdot (v_{sink}+v_{steig})
	\end{equation}
	und der Fehler
	\begin{equation}
		\Delta Q = Q \cdot \sqrt{(\frac{\Delta \eta}{\eta})^2 + (\frac{\Delta d}{d})^2+(\frac{\Delta r}{r})^2 + \frac{\Delta v_{sink}^2 + \Delta v_{steig}^2}{(v_{sink} + v_{steig})^2}}
	\end{equation}
	damit ergab sich die folgende Tabelle 
	\begin{figure}[H]
		\begin{tabular}{c|c|c|c|c|c|c|c}
			\hline
			$v_{sink} [\mu m/s]$ & $\Delta v_{sink} [\mu m/s]$ & $v_{steig} [\mu m/s]$ &$\Delta v_{steig} [\mu m/s]$ & $r [\mu m] $&$ \Delta r [\mu m]$ &$ Q [10^{-9} C] $&$ \Delta Q [10^{-9} C]$\\
			\hline
			212.9 & 5.5 & 121.9 & 1.8 & 0.61 & 0.019 & 2.1 & 0.083\\
			295.5 & 4.8 & 238.6 & 4.0 & 0.48 & 0.026 & 2.6 & 0.154\\
			196.0 & 2.1 & 70.4 & 0.9 & 0.71 & 0.008 & 1.9 & 0.042\\
			193.9 & 2.5 & 70.2 & 1.0 & 0.71 & 0.009 & 1.9 & 0.044\\
			206.5 & 2.3 & 100 & 1.1 & 0.66 & 0.009 & 2.1 & 0.046\\
			220.6 & 2.6 & 129.6 & 1.4 & 0.61 & 0.010 & 2.2 & 0.054\\
			297.7 & 3.4 & 98.3 & 1.6 & 0.90 & 0.010 & 3.6 & 0.080\\
			212.6 & 2.7 & 120.4 & 1.4 & 0.61 & 0.010 & 2.1 & 0.053\\
			304.0 & 5.5 & 252.6 & 3.6 & 0.46 & 0.029 & 2.6 & 0.174\\
			198.2 & 2.1 & 74.4 & 0.9 & 0.71 & 0.008 & 2.0 & 0.042\\
			196.5 & 2.1 & 81.3 & 0.9 & 0.68 & 0.008 & 1.9 & 0.042\\
			377.1 & 5.4 & 123.1 & 1.7 & 1.01 & 0.013 & 5.2 & 0.122\\
			211.8 & 2.6 & 123.8 & 2.1 & 0.60 & 0.012 & 2.0 & 0.056\\
			199.5 & 3.2 & 55.6 & 1.0 & 0.76 & 0.010 & 2.0 & 0.049\\
			202.6 & 2.2 & 89.4 & 1.0 & 0.68 & 0.008 & 2.0 & 0.044\\
			199.5 & 2.6 & 81.0 & 1.0 & 0.69 & 0.009 & 2.0 & 0.045\\
			302.9 & 3.4 & 144 & 1.7 & 0.80 & 0.010 & 3.7 & 0.083\\
			301.1 & 3.8 & 73.7 & 1.3 & 0.96 & 0.010 & 3.7 & 0.081\\
			204.1 & 3.9 & 120.2 & 3.0 & 0.58 & 0.017 & 1.9 & 0.072\\
			299.4 & 4.2 & 262.2 & 3.0 & 0.39 & 0.026 & 2.2 & 0.160\\
		\end{tabular}
	\end{figure}
	Trägt man nun die unkorrigierten Radien gegen deren zugehörige Ladung auf ergbibt sich folgende Grafik
	\begin{figure}[H]
		\centering
		\includegraphics[width=1.0\textwidth]{QoverRUncorr.pdf}
		\caption{Ladung als Funktion des Radius}
	\end{figure}
	Es ist deutlich zu erkennen, dass eine Quantifizierung Wahrscheinlich ist, jedoch stimmt diese
	nicht mit den zu erwartenden Werten über ein, weswegen man die Cunningham Korrektur zu Hilfe nimmt.
	Mit der Cunningham Korrektur ergeben sich die Radien und Ladungen über
	\begin{equation}
		r_c = \sqrt{(\frac{A_c <l>}{2})^2 + r^2} - \frac{A_c <l>}{2} 
	\end{equation}
	\begin{equation}
		Q_c = Q(1+\frac{A_c<l>}{r_c})^{-1} \frac{r_c}{r}
	\end{equation}
	mit r als unkorrigierter Radius, $A_c=1.26$ als einheitenloser Faktor und $<l>=6.4\cdot 10^{-8} m$ als
	mitlere freie Weglänge eines Teilchens in Luft und den Fehlern
	\begin{equation}
		\Delta r_c = \frac{r \Delta r}{\sqrt{(\frac{A_c <l>}{2})^2 + r^2}}
	\end{equation}
	\begin{equation}
		\Delta Q_c = Q_c \sqrt{(\frac{\Delta Q}{Q})^2 + (\frac{A_c <l> \Delta r}{r(r+<l>A_c)})^2}
	\end{equation}
	Dies lässt sich erneut in folgender Tabelle darstellen und Plotten
	\begin{figure}[H]
		\begin{tabular}{c| c| c| c| c| c| c| c}
			\hline
			$r [\mu m]$ & $\Delta r [\mu m] $& $Q [10^{-9} C]$ & $\Delta Q [10^{-9} C]$ & $r_c [\mu m]$ & $\Delta r_c [\mu m]$ & $Q_c [10^{-9} C]$& $\Delta Q_c [10^{-9} C]$\\
			\hline
			0.61 & 0.019 & 2.1 & 0.08 & 0.57 & 0.019 & 1.72 & 0.083\\
			0.48 & 0.026 & 2.6 & 0.15 & 0.44 & 0.026 & 2.07 & 0.155\\
			0.71 & 0.008 & 1.9 & 0.04 & 0.67 & 0.008 & 1.65 & 0.042\\
			0.71 & 0.009 & 1.9 & 0.04 & 0.67 & 0.009 & 1.62 & 0.044\\
			0.66 & 0.008 & 2.1 & 0.05 & 0.62 & 0.009 & 1.73 & 0.046\\
			0.61 & 0.010 & 2.2 & 0.05 & 0.57 & 0.010 & 1.80 & 0.054\\
			0.90 & 0.010 & 3.6 & 0.08 & 0.86 & 0.010 & 3.20 & 0.080\\
			0.61 & 0.010 & 2.1 & 0.05 & 0.57 & 0.010 & 1.72 & 0.053\\
			0.46 & 0.030 & 2.6 & 0.17 & 0.42 & 0.029 & 2.02 & 0.176\\
			0.71 & 0.010 & 2.0 & 0.04 & 0.67 & 0.008 & 1.68 & 0.042\\
			0.80 & 0.010 & 2.0 & 0.04 & 0.64 & 0.008 & 1.64 & 0.042\\
			1.01 & 0.013 & 5.2 & 0.12 & 0.97 & 0.013 & 4.62 & 0.122\\
			0.60 & 0.012 & 2.0 & 0.06 & 0.56 & 0.012 & 1.69 & 0.056\\
			0.76 & 0.010 & 2.0 & 0.05 & 0.72 & 0.010 & 1.71 & 0.049\\
			0.68 & 0.010 & 2.0 & 0.04 & 0.64 & 0.008 & 1.70 & 0.045\\
			0.69 & 0.010 & 2.0 & 0.05 & 0.65 & 0.009 & 1.68 & 0.046\\
			0.80 & 0.010 & 3.7 & 0.08 & 0.76 & 0.010 & 3.17 & 0.083\\
			0.96 & 0.010 & 3.7 & 0.08 & 0.92 & 0.010 & 3.26 & 0.081\\
			0.58 & 0.017 & 1.9 & 0.07 & 0.54 & 0.017 & 1.59 & 0.072\\
			0.39 & 0.027 & 2.2 & 0.16 & 0.35 & 0.027 & 1.66 & 0.162\\
	   \end{tabular}
	\end{figure}		
	\begin{figure}[H]
		\centering
		\includegraphics[width=1.0\textwidth]{QoverRUncorrCunt.pdf}
		\caption{$Q_c$ als Funktion von $r_c$}
	\end{figure}
	In dieser Grafik ist eine deutlich bessere Übereinstimmung der Messwerte mit der Quantisierten Elementarladung zu
	erkennen. Einige der Messwerte fallen leider nicht innerhalb des Fehlerbereichs mit einem Vielfachen der Elementarladung
	zusammen, diese Abweichung lässt sich für uns durch Ungenauigkeiten in der Messung zurückführen, wie zum beispiel
	die Umkehr der Kondersatorspannung und der daraus resultierenden kurzen Beschleunigungsstrecke, sowie die Bewegung der
	Teilchen in 3D, die Brownsche Molekularbewegung und die Umweltbedingte Bewegung der Luft im Kondensator.
	
	Zuletzt lässt sich nicht desto trotz die Elementarladung aus den Messwerten bestimmen. Dafür gilt
	\begin{equation}
		e = \frac{1}{\sum_{i=1}^n w_i} \sum_{i=1}^n Q_i \cdot w_i
	\end{equation}
	mit 
	\begin{equation}
		w_i = \frac{1}{\Delta Q_i^2}
	\end{equation}
	und dem Fehler
	\begin{equation}
		\Delta e = \sqrt{\sum_{i=1}^n\frac{1}{w_i}}
	\end{equation}
	Woraus sich für unsere Messung eine Elementarladung von $(1,83\pm 0.39)\cdot 10^{-19} C$ ergibt.
	Diese liegt innerhalb des Fehlerbereichs auf dem Literaturwert von $e=1.602176634\cdot 10^{-19}C$.
\section{Diskussion}
	Abschließend noch eine Zusammenfassung. Die unkorrigierten Messwerte stimmen, wie zu erwarten, 
	nicht mit den Literaturwerten über ein. Die korrigierten jedoch passen schon besser. 
	Auffällig ist, dass die korrigierten Ladungen und Radien geringer sind, als die ursprünglichen. 
	Dies war zu erwarten, da die Korrektur die Geschwindigkeiten nach unten korrigiert. 

	Der große Fehler bei unserem Ergebnis lässt sich dadurch erklären, dass bei der Berechnung
	der Elementarladung alle Messwerte eingeflossen sind. Bei Betrachtung der Grafiken wird jedoch deutlich,
	dass ein paar der Messwerte wahrscheinlich Mehrfachgeladen sind. Der Großteil ist aber Einfach geladen.
	Einige andere Fehlerquellen wären, wie oben schon kurz angeschnitten, dass die Teilchen auch einen Drift 
	aus der Bildebene raus hatten. Was dazu führt, dass die Tröpchen während des Messvorgangs größer/kleiner und
	bei einer zu späten Anpassung des Mikroskops unscharf wurden.

	Insgesamt hätte eine Statistik mit mehr Messwerten zu einem besseren Ergebnis geführt, da die Mehrfachladungen
	seltener sind, als die einfach geladenen Tröpchen. Jedoch ist auch unser Ergebnis mit 20 Tröpchen auf
	ein Ergebnis gekommen, welches mit Fehlerbereichs auf dem Literaturwert liegt.

\end{document} 