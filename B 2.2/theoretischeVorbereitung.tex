\section{Theoretische Vorbereitung}
    \subsection{Reziprokes Gitter}
        Das reziproke Gitter beschreibt in der Festkörperphysik die Röntgen-, Elektronen-, und Neutronenbeugung
        an Kristallinen strukturen. Es wird häufig in zusammenhang mit den Miller'schen Indizes verwendet
        um die Netebenen $(hkl)$ zu beschreiben. Es bietet sich an diese im Reziproken zu definieren, da die Länge
        eines Vektors der die Position eines Gitterpunkts beschreibt gleich dem Reziproken des Abstands der
        Netzebenen entspricht.
        Aus den Basisvektoren des Punktgitter ($\vec{a_1},\vec{a_2},\vec{a_3}$) ergeben sich über folgende Beziehung
        die Basisvektoren ($\vec{b_1},\vec{b_2},\vec{b_3}$) des Reziproken gitters.
        \begin{align*}
            \vec{b_1} = 2\pi \frac{\vec{a_2}\times \vec{a_3}}{\vec{a_1}\cdot (\vec{a_2}\times \vec{a_3})}
            \\\vec{b_2} = 2\pi \frac{\vec{a_3}\times \vec{a_1}}{\vec{a_1}\cdot (\vec{a_2}\times \vec{a_3})}
            \\\vec{b_3} = 2\pi \frac{\vec{a_1}\times \vec{a_2}}{\vec{a_1}\cdot (\vec{a_2}\times \vec{a_3})}
        \end{align*}
        Über dieses Definition der Basisvektoren lassen sich die Koordinateneines Punktes im reziproken Gitter
        über die Miller'schen indizes $(hkl)$ beschreiben.
        
        \subsubsection*{Bragg Gleichung}
            Die Bragg Gleichung liefert einen Zusammenhang zwischen dem Netzebenenabstand $d_{hkl}$ und dem
            Beugungswinkel $\theta$. Damit dieser Zusammenhang gilt muss jedoch der einfallende und gestreute
            Strahl symetrisch zur reflektierende Netzebene verlaufen. Dann lässt sich der Zusammenhang beschreiben durch
            \begin{equation}
                n\lambda = 2d_{hkl} \sin(\theta)
            \end{equation}
            aus dieser lässt sich die äquivalente Laue Bedingung ableiten, welche aussagt,
            dass ein Röntgenstrahl genau dann gestreut wird, wenn der Beugungsvektor $\vec{k}$ gleich dem
            reziproken Gittervektor ist. 
    
    \subsection{Ordnungsparameter und Phasenübergänge}
       Bei einem Phasenübergang handelt es sich um eine Umwandelung einer Phase eines Stoffes in 
       eine andere Phase. Diese Übergänge treten meist in Abhängigkeit von einem oder mehrerer Zustandsvariablen
       wie Druck oder Temperatur auf.\\
       Will man nun den Zustand eines Physikalischen systems nicht nur vor und nach einem Übergang beschreiben,
       so dienen die Ordnungsparameter zur eben dieser. Geht man beispielsweise von einem Übergang von einer
       flüssigen in eine feste Phase, wie beispielweise bei gefrierung von Wasser, so geht das System von einer
       hohen Symetrie in eine Phase in der lediglich die Gittersymetrie verbleibt. Dieser Übergang lässt sich
       anhand des Ordnungsparameters als Übergang von absoluter Unordnung ($s=0$) zu einer höheren Ordnung
       ($s=c\in \Re^+$) beschreiben. Diese Beschreibung lässt sich auf beliebige Übergänge übertragen, bei denen
       gegebenenfalls kein eindeutiger Phasenwechsel auftritt, ja nach dem verändert sich der Ordnungsparameter
       entweder plötzlich oder kontinuierlich. Anhand der Thermondynamik lässt sich über
       \begin{equation}
           F = E-TS
       \end{equation}
       zeigen, dass der Ordnungsparameter stehts versucht die freie Energie zu minimieren um schlussendlich
       einen Gleichgewichtszustand mit mininmaler freien Energie zu erreichen. 

    \subsection{Überstrukturen}
        Eine Überstruktur beschreibt eine Elementarzelle die größer ist als diejenige die man beim Durchschneiden
        des Kristallgitters erhalten würde. Nimmt man beispielsweise eine reine Oberfläche/Kristalline Struktur an, so
        gäbe es keine Überstrukturen, diese kommen erst dann zustande wenn beispielweise Adsorbatome an einer Oberfläche
        eine weiteres geordnetes Gitter bilden welches größer ist als das des reinen ursprünglichen Gitters.
        Überstrukturen werden nach Wood, über ein vielfaches der reziproken Gittervektoren angegeben, beispielweise
        \begin{align*}
            (2\times1) \text{  die Überstruktur ist in x-Richtung doppelt so groß wie die Elementar Zelle}\\
            (\sqrt{2}\times\sqrt{2})R45 \text{  um 45° rotierte quadratische Zelle}
        \end{align*}
        Überstrukturen lassen sich beispielsweise direkt mit dem Rastertunnelmikroskop sichtbar gemacht werden.
        Andererseits lässt sich über Beugungsverfahren das reziproke Gitter der Oberfläche abbilden, bei der die
        Überstruktur zu zusätzlichen Gitterpunkten im reziproken Gitter in form von zusätzlichen Beugungsmaxima
        führt. 
        \subsubsection{$CuZn$}
            
        \subsubsection{$CuAu$}

        \subsubsection{$Cu_3Au$}

    \subsection{Die röntgenographische Methode}

        \subsubsection{Aufbau eines Röntgendiffraktometers}   

        \subsubsection{Röntgenstrahlung}

        \subsubsection{Intensität der gestreuten Röntgenstrahlung}

        \subsection{Reflexindizierung im Röntgendiffraktogramm}
    
    \subsection{Die resistive Methode}
        
