\section{Theoretische Vorbereitung}
    \subsection{Reziprokes Gitter}
        Das reziproke Gitter beschreibt in der Festkörperphysik die Röntgen-, Elektronen-, und Neutronenbeugung
        an Kristallinen strukturen. Es wird häufig in zusammenhang mit den Miller'schen Indizes verwendet
        um die Netebenen $(hkl)$ zu beschreiben. Es bietet sich an diese im Reziproken zu definieren, da die Länge
        eines Vektors der die Position eines Gitterpunkts beschreibt gleich dem Reziproken des Abstands der
        Netzebenen entspricht.
        Aus den Basisvektoren des Punktgitter ($\vec{a_1},\vec{a_2},\vec{a_3}$) ergeben sich über folgende Beziehung
        die Basisvektoren ($\vec{b_1},\vec{b_2},\vec{b_3}$) des Reziproken gitters.
        \begin{align*}
            \vec{b_1} = 2\pi \frac{\vec{a_2}\times \vec{a_3}}{\vec{a_1}\cdot (\vec{a_2}\times \vec{a_3})}
            \\\vec{b_2} = 2\pi \frac{\vec{a_3}\times \vec{a_1}}{\vec{a_1}\cdot (\vec{a_2}\times \vec{a_3})}
            \\\vec{b_3} = 2\pi \frac{\vec{a_1}\times \vec{a_2}}{\vec{a_1}\cdot (\vec{a_2}\times \vec{a_3})}
        \end{align*}
        Über dieses Definition der Basisvektoren lassen sich die Koordinateneines Punktes im reziproken Gitter
        über die Miller'schen indizes $(hkl)$ beschreiben.
        
        \subsubsection*{Bragg Gleichung}
            Die Bragg Gleichung liefert einen Zusammenhang zwischen dem Netzebenenabstand $d_{hkl}$ und dem
            Beugungswinkel $\theta$. Damit dieser Zusammenhang gilt muss jedoch der einfallende und gestreute
            Strahl symetrisch zur reflektierende Netzebene verlaufen. Dann lässt sich der Zusammenhang beschreiben durch
            \begin{equation}
                n\lambda = 2d_{hkl} \sin(\theta)
            \end{equation}
            aus dieser lässt sich die äquivalente Laue Bedingung ableiten, welche aussagt,
            dass ein Röntgenstrahl genau dann gestreut wird, wenn der Beugungsvektor $\vec{k}$ gleich dem
            reziproken Gittervektor ist. 
    
    \subsection{Ordnungsparameter und Phasenübergänge}
            

    \subsection{Überstrukturen}
        
        \subsubsection{$CuZn$}

        \subsubsection{$CuAu$}

        \subsubsection{$Cu_3Au$}

    \subsection{Die röntgenographische Methode}

        \subsubsection{Aufbau eines Röntgendiffraktometers}   

        \subsubsection{Röntgenstrahlung}

        \subsubsection{Intensität der gestreuten Röntgenstrahlung}

        \subsection{Reflexindizierung im Röntgendiffraktogramm}
    
    \subsection{Die resistive Methode}
        
