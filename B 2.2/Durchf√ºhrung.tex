\section{Durchführung}
    Es werden drei $Cu_3Au$ Proben auf ihre Ordnungsparameter untersucht. Um einen unterschiedlichen Ordnungsparameter
    zu erzeugen, wurden die drei Proben auf über 500 °C ($>T_C = 386 °C$) erhitzt und danach unterschiedlich abgekühlt.
    Um eine vollständige Ordnung zu erzeugen, wird eine der Proben sehr langsam abgekühlt. Für Unordnung wird eine Probe
    in Wasser abgeschreckt. Die teilweise Ordnung wird durch ca. zweistündiges halten bei 370 °C mit anschließendem abkühlen
    erreicht. Von jeder dieser Proben wird nun ein Röntgendiffraktogramm aufgenommen, sowie der elektrische Widerstand 
    mittels der Vierpunktmethode bestimmt. Daraus lassen sich dann die Ordnungsparameter der Proben bestimmen.