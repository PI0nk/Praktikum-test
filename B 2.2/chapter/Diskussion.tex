\section{Diskussion}
    Abschließend noch eine Zusammenfassung des Versuchs und eine Diskussion
    unserer Ergebnisse.
    Ziel des Versuchs war es, von drei Proben von Cu$_3$Au mit einem
    unterschiedlichen Ordnungsgrad, diesen zu bestimmen. 
    Dies wurde mit zwei verschiedenen Messmethoden erreicht: 
    \begin{itemize}
        \item Der Röntgendiffraktometrie
        \item dem resistiven Verfahren
    \end{itemize}
    Bei der Röntgendiffraktometrie wurde sich die Bragg-Bedingung zunutze gemacht
    indem man die Fundamentalreflexe mit den Überstrukturreflexen verglichen und 
    daraus der Ordnungsgrad bestimmen konnte, sowie auch die Gitterkonstante
    von Cu$_3$Au. Bei dem resistiven Verfahren wurden die Ordnungsgrade 
    durch Widerstandsmessung bei kleinen Temperaturen bestimmt.

    Unsere Ergebnisse für die Gitterkonstanten der 3 Proben $\bar{a}_{Probe2} = (3,74 \pm 0,002)[\mathring{A}]$,
    $\bar{a}_{Probe3} = (3,77 \pm 0,006)[\mathring{A}]$ und $\bar{a}_{Probe4} = (3,86 \pm 0,001)[\mathring{A}]$ liegen
    alle nicht im Fehlerbereich des Literaturwerts: $a_{Cu_3Au} = 3.7490\mathring{A}$ \footnote{http://som.web.cmu.edu/structures/S005-Cu3Au.html}
    jedoch ist eine Abweichung von 0,2 \% sehr gut (Probe 2), bei Probe 4 ist die Abweichung schon 3\%, was aber 
    auch noch sehr zufriedenstellen ist. Eine mögliche Abweichung der Proben könnte daher kommen, dass das Messgerät bei den verschieden Proben
    unterschiedlich justiert wurde. Außerdem war es schwierig und relativ ungenau die Überstrukturpeaks zu bestimmen und zu fitten.
    Die Fundamentalreflexe waren bei jeder Probe deutlich ausgeprägt und leicht zu identifizieren, die Überstrukturpeaks 
    waren jedoch teilweise kaum vom Rauschen zu unterscheiden.
    
    Bei der Bestimmung des Ordnungsgrades der beiden Proben mit der röntgenographischen Methode kam heraus, 
    das Probe 2 um ca. 18\% geordneter ist, als Probe 3. Jedoch ist der Fehler des Ordnungsgrades von Probe 3 sehr groß.
    Dieser große Fehler wird durch den einen fehlerhaften Messwert bzw. errechneten Ordnungsgrad $>1$ verursacht.
    Eine mögliche Erklärung für diesen Wert ist, wie oben schon beschrieben, die schwer zu erkennenden Überstrukturpeaks.
    Das Probe 4 vollständig ungeordnet ist, war schon klar als nur die Fundamentalreflexe aufgetreten sind. Dies wurde
    auch bei der resistiven Methode erkannt. Alles in allem gehen unsere Ergebnisse des ersten Veruchsteils
    aber in Ordnung.

    Bei der Auswertung der resistiven Methode haben wir angenommen, dass die Probe 4 vollständig
    ungeordnet und die Probe 2 vollständig geordnet ist. Dadruch konnten wir dann, durch umstellen einiger Formeln
    auf ein Ordnungsgrad der Probe 3 von $S_{Probe 3} = 0.997 \pm 0.018$ errechnen. 
    Dies würde einer Probe, welche nahezu vollständig geordnet ist, entsprechen. 
    Der Literaturwert für den speziefischen Widerstand einer geordneten Cu$_3$Au Probe beträgt: $4.2 \cdot 10^{-8}\Omega$, von einer ungeordneten
    Probe: $11,4\cdot 10^{-8}\Omega$ \footnote{https://onlinelibrary.wiley.com/doi/epdf/10.1002/andp.19334100504}.
    Unser errechneten Werte der spezifischen Widerstände weichen um einiges an dem Literaturwert ab.
    Probe 2 um 40\%, Probe 3 um 50\% und Probe 4 um 70\%. Diese Abweichungen lassen sich durch 
    unsere vorher getroffenen Annahmen erklären. Anscheinend sind weder Probe 2 noch Probe 4 vollständig geordnet bzw. ungeordnet.
    Dies ist eine mögliche Fehlerquelle, auch für unsere Bestimmung des Ordnungsgrades.
    
    Alles in allem lässt sich sagen, dass wir nur für die Probe 4 festellen konnten, dass diese vollständig ungeordnet sein muss
    (nur Fundamentalreflexe und größter Widerstand). Für die Proben 2 und 3 konnten wir jedoch keine wirklich zufrieden stellenden
    Ordnungsgrade bestimmen. Laut Versuchsteil 1 sind die Proben weder vollständig geordnet noch ungeordnet, sondern in einem
    zwischen Zustand. In Versuchsteil 2 jedoch nehmen wir Probe 2 als vollständig geordnet an und bestimmen
    dann einen Ordnungsgrad von $0.997$ für Probe 3, was nahezu vollständig geordnet entspricht. 
    Unsere bestimmten Gitterkonstanten sind jedoch sehr genau und sehr zufriedenstellend.