\subsection{resistives Verfahren}
    Um in diesem Abschnitt den Ordnungsparameter bestimmen zu können wurde zunächst der Widerstand der Verschiedener Proben mit
    unterschiedlichem Ordnungsparameter $S$ gemessen. Um daraus den Ordnungsparameter der Proben zu bestimmen
    wurden die Daten zunächst durch ein lineares lineare abhängigkeit des Widerstands von der Temperatur
    approximiert, welches for $T\Rightarrow 0$ in einen konstanten Restwiderstand übergeht.
    \begin{figure}[H]
        \centering
        \includegraphics[width=0.8\textwidth]{Messdaten/Auswertungsskripte/Widerstand.pdf}
        \caption{Plot der Gemessenen Widerstände in abhängigkeit von der Temperatur, in Rot das gefittete Modell, links von der Grauen linie wird
        ein konstantes Restwiderstandsverhalten angenommen, rechts davon ein lineares Wachstum}
        \label{resistance plot}
    \end{figure}
    Für den Widerstand einer Legierung kann für den Restwiderstand folgendes Verhalten angenommen werden
    \begin{equation}
        \rho_D(x) = \rho_D(T=0) + A\cdot x(1-x)(1-S^2)
    \end{equation}
    wobei $A$ eine zu bestimmende Materialkonstante ist.\\
    Um $A$ zu bestimmen wurde zunächst angenommen, das aufgrund der Gitterstruktur die Probe mit dem höchsten
    Restwiderstand ebenfalls die größste Unordnung aufweißt, dort wurde $S=0$ angenommen. Ebenso wurde angenommen
    das die Probe mit dem geringsten Restwiderstand die geordneste ist. Dort gilt $S=1$. Diese annahmen sind
    notwendig um die Materialkonstante aus den Daten zu ermitteln, da sonst unser Gleichungssystem unterbestimmt wäre,
    jedoch sollten alle daraus resultierenden ergebnisse mit vorsicht betrachtete werden, da sowohl ein Ordnungsparameter
    von ~1 als auch von ~0 äussert schwer zu erreichen sind. Nutzt man jedoch diese annahme, lässt sich die Materialkonstante
    mithilfe der gefitteten modelle bestimmen. Dazu wird angenommen, das es sich bei den Proben um $Cu_3Au$ handelt, womit
    aus der Anleitung über $Cu_{1-x}Aux$ ein anteil von 75\% Kupfer und 25\% Gold, und damit $x=0.25$ folgt. Damit folgt weiter
    \begin{equation}
        \rho_{D_{max}}(0.25) = \rho_D(T=0) + A\cdot 0.25(1-0.25)(1-S^2)
    \end{equation} 
    $\rho_D(T=0)$ gibt hierbei einen Konstanten Restwiderstand einer reinen Probe bei $0K$ an.
    Ferner folgt
    \begin{align*}
        \rho_{(S=1)} = \rho_D(T=0)\\
        \rho_{(S=0)} = \rho_D(T=0) + Ax(1-x)
    \end{align*}
    \begin{equation}
        A=\frac{\rho_{(S=0)}-\rho_{(S=1)}}{x(1-x)}
    \end{equation}
    \begin{equation}
        \Delta A=\frac{1}{x(1-x)}\sqrt{\Delta \rho_{(S=0)}^2+ \Delta \rho_{(S=1)}^2}
    \end{equation}
    \begin{equation}
        S = \sqrt{\frac{\rho_{(S=0)}-\rho_{(S=?)}}{\rho_{(S=0)}-\rho_{(S=1)}}}
    \end{equation}
    \begin{equation}
        \Delta S = \sqrt{(\Delta \rho_{(S=0)} \frac{\rho_{(S=?)}-\rho_{(S=1)}}{2(\rho_{(S=0)}-\rho_{(S=1)})^2 S})^2 + ( \frac{\Delta \rho_{(S=?)}}{2(\rho_{(S=0)}-\rho_{(S=1)}) S})^2 + (\Delta \rho_{(S=1)} \frac{(\rho_{(S=0)}-\rho_{(S=?)})}{2(\rho_{(S=0)}-\rho_{(S=1)})^2 S})^2}
    \end{equation}
    Um nun S bestimmen zu können muss zunächst aus den Wiederstandswerten zusammen mit den geometrien
    der Proben der zugehörige Spezifische Widerstand bestimmt werden. Dafür gilt folgende relation:
    \begin{equation}
        \rho = \frac{b\cdot d \cdot R}{l}
    \end{equation}
    \begin{equation}
        \Delta \rho = \sqrt{(\frac{d R \Delta b}{l})^2 + (\frac{b R \Delta d}{l})^2 + (\frac{db\Delta R}{l})^2 + (\frac{db R \Delta l}{l^2})^2}
    \end{equation}
    für die probengeometrien
    \begin{figure}[H]
        \centering
        \begin{tabular}{l|c|c|c}
             & Länge l [mm] & Breite b [mm] & Dicke d [mm] \\
            \hline
            Probe 2 & $6.7\pm 0.05$ & $5.1\pm 0.03$ & $0.2\pm 0.05$ \\
            Probe 3 & $4.3\pm 0.05$ & $5.2\pm 0.03$ & $0.2\pm 0.05$ \\
            Probe 4 & $4.1\pm 0.05$ & $5.1\pm 0.03$ & $0.2\pm 0.05$ \\
        \end{tabular}
    \end{figure}
    und damit folgt
    \begin{figure}[H]
        \centering
        \includegraphics[width=0.8\textwidth]{Messdaten/Auswertungsskripte/spezWiderstand.pdf}
        \caption{Plot der Spez Widerstände}
        \label{spec resistance plot}
    \end{figure}
    aus der Modellbetrachtung folgen Restwiderstände von
    \begin{figure}[H]
        \centering
        \begin{tabular}{c|c|c}
            Probe & Widerstand R [$\Omega$] &spezifischer Widerstand [$\Omega m$] \\
            \hline
            Probe 2 & $(3.9\pm 0.0099)e-4 $ & $(5.9\pm 1.5)e-08$ \\
            Probe 3 & $(2.6\pm 0.015)e-4 $ & $(6.3\pm 1.6)e-08$ \\
            Probe 4 & $(2.6\pm 0.00091)e-3 $ & $(6.5\pm 1.6)e-07$ \\
        \end{tabular}
    \end{figure}
    Darauf folgt entsprechend $A=(7.85\pm8.7)e-7 \Omega m$ und damit für Probe 3 auch $S=0.997 \pm 0.018$. Dieses Resultat
    entspricht unseren erwartungen, da Probe 3 und 2 etwa den selben Restwiderstand bei T=5K besitzen.