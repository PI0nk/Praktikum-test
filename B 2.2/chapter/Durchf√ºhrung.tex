\section{Durchführung}
    Es werden drei $Cu_3Au$ Proben auf ihre Ordnungsparameter untersucht. Um einen unterschiedlichen
    Ordnungsparameter zu erzeugen, wurden die drei Proben auf über 500 °C ($>T_C = 386 °C$) erhitzt
    und danach unterschiedlich abgekühlt. Um eine vollständige Ordnung zu erzeugen, wird eine der
    Proben sehr langsam abgekühlt. Für Unordnung wird eine Probe in Wasser abgeschreckt.
    Die teilweise Ordnung wird durch ca. zweistündiges halten bei 370 °C mit anschließendem abkühlen
    erreicht. Von jeder dieser Proben wird nun ein Röntgendiffraktogramm aufgenommen,
    sowie der elektrische Widerstand mittels der Vierpunktmethode bestimmtum anschließend mit beiden Verfahren den
    Ordnungsparameter der Proben zu ermitteln.
    Bei dieser Durchführung ist jedoch auf ein paar Punkte zu achten um bei der Auswertung die Bestmöglichen
    Ergebnisse zu erhalten.\\
    Die zu Röntgendiffraktometrie genutzte Röntgenröhre emittiert $K_{\alpha}$ Strahlung mit bekannter
    Wellenlänge, diese ist jedoch nicht eindeutig, da durch die Spin-Bahn-Kopplung zwei Verschiedene
    Relaxionsenergien Auftreten können woraus Röntgenstrahlung verschiedener Wellenlänge auftritt.
    Diese muss zunächst Charakterisiert werden (hier gegeben) um den Einfluss des Linienspektrums auf
    das Diffraktogramm zu bestimmen. In diesem Fall lassen sich die Wellenlängen $\lambda_1 = 1.5406\mathring{A}$
    und $\lambda_2 = 1.5444\mathring{A}$ über ihr Verhältniss $\frac{K_2}{K_1} = 0.52$ mitteln um mögliche Doppelpeaks
    zu erkennen und interpretieren zu können.\\
    Desweiteren ist bei der Vierpunktmethode für die Widerstandsmessung darauf zu achten das der Strom alle paar
    Messpunkte umgepolt wird um die Messungenauigkeit des Messgeräts durch anschließende Mittelung der 
    Messwerte negieren zu können. Ebenfalss ist darauf zu achten, das die Probengeometrie bzw. die Maße
    der Probe bekannt sind, da der spezifische Wiederstand sich wie folgt aus dem gemessenen Flächenwiderstand
    bestimmen lässt
    \begin{equation}
        R_{sq} = \frac{\rho \cdot l}{d\cdot b}
    \end{equation}.