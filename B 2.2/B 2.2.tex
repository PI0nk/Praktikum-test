\documentclass{article}

\usepackage[ngerman]{babel}
\usepackage{pdfpages}
\usepackage{amssymb}
\usepackage{amsmath}
\usepackage{mathtools}
\usepackage{graphics}
\usepackage{graphicx}
\usepackage{geometry}
\usepackage{float} 
\usepackage{color,soul}

%\usepackage[demo]{graphicx}
\usepackage{caption}
\usepackage{subcaption}

\geometry{
 a4paper,
 total={170mm,257mm}, 
 left=25mm,
 top=25mm, 
}
 

\begin{document} 
 
\thispagestyle{empty}
\vspace*{\fill}
\begin{center}
	\Huge
	\textbf{Universität zu Köln}\\
	\LARGE
	\textbf{Institut für Festkörperphysik}\\
	\vspace{2cm}
	\textbf{Versuchsprotokoll}\\   
	\vspace{0.5cm}
	\large
	\textbf{B2.2: Überstruktur in $Cu_3Au$}\\
	\normalsize
	\vspace{2cm}
	\begin{tabular}{r l}
		Autoren: 	& Jesco Talies$^1$\\
					& Timon Danowski$^2$\\
		Durchgefuehrt am:	& 19.05.2021\\
		Betreuer:	& Julian Wagner
	\end{tabular}
\end{center}
\vfill\footnotesize
$^1$ jtalies@smail.uni-koeln.de, Matrikel-Nr.: 7348338\\ 
$^2$ tdanowsk@smail.uni-koeln.de, Matrikel-Nr.: 7348629
\normalsize 

\newpage
\thispagestyle{empty}
\tableofcontents
\clearpage
\setcounter{page}{1}
\section{Einleitung}
    In vielen Legierungen bildet sich zusätzlich zu der Gitterstruktur des Festkörpers eine übergeordnete
    Struktur, die sogenannte Überstruktur. Sie lässt sich in vergleichsweise Makroskopischen Systemen
    über die Minimierung der Energie erreichen und ist häufig beeinflusst durch Fehlstellen und Deformationen.
    Diese Überstrukturen lassen sich beeinflussen bzw. erzeugen, sie treten nur unterhalb einer kritischen
    Temperatur auf, sodass sich durch gezieltes Erhitzen und Abkühlen eines Systems, Proben mit mehr
    oder Weniger Ordnung erzeugen lassen, sodass im resultierenden Spektrum die Unterschiede zu erkennen sind.
    Im folgenden Versuch werden wir uns genau dieses Phänomen zu nutze machen, indem drei verschieden geordnete
    Proben miteinander vergleichen werden. Dazu wird zunächst die röntgenographische Methode und anschließend
    die restive verwendet.

    \begin{figure}[H]
        \centering
        \includegraphics{images/einleitung_hurensohn.jpg}
        \label{einleitung}
        \caption{Basiszelle Cu$_3$Au}
    \end{figure}
   
\section{Theoretische Vorbereitung}
    \subsection{Reziprokes Gitter}
        Das reziproke Gitter beschreibt in der Festkörperphysik die Röntgen-, Elektronen-, und Neutronenbeugung
        an Kristallinen strukturen. Es wird häufig in zusammenhang mit den Miller'schen Indizes verwendet
        um die Netebenen $(hkl)$ zu beschreiben. Es bietet sich an diese im Reziproken zu definieren, da die Länge
        eines Vektors der die Position eines Gitterpunkts beschreibt gleich dem Reziproken des Abstands der
        Netzebenen entspricht.
        Aus den Basisvektoren des Punktgitter ($\vec{a_1},\vec{a_2},\vec{a_3}$) ergeben sich über folgende Beziehung
        die Basisvektoren ($\vec{b_1},\vec{b_2},\vec{b_3}$) des Reziproken gitters.
        \begin{align*}
            \vec{b_1} = 2\pi \frac{\vec{a_2}\times \vec{a_3}}{\vec{a_1}\cdot (\vec{a_2}\times \vec{a_3})}
            \\\vec{b_2} = 2\pi \frac{\vec{a_3}\times \vec{a_1}}{\vec{a_1}\cdot (\vec{a_2}\times \vec{a_3})}
            \\\vec{b_3} = 2\pi \frac{\vec{a_1}\times \vec{a_2}}{\vec{a_1}\cdot (\vec{a_2}\times \vec{a_3})}
        \end{align*}
        Über dieses Definition der Basisvektoren lassen sich die Koordinateneines Punktes im reziproken Gitter
        über die Miller'schen indizes $(hkl)$ beschreiben.
        
        \subsubsection*{Bragg Gleichung}
            Die Bragg Gleichung liefert einen Zusammenhang zwischen dem Netzebenenabstand $d_{hkl}$ und dem
            Beugungswinkel $\theta$. Damit dieser Zusammenhang gilt muss jedoch der einfallende und gestreute
            Strahl symetrisch zur reflektierende Netzebene verlaufen. Dann lässt sich der Zusammenhang beschreiben durch
            \begin{equation}
                n\lambda = 2d_{hkl} \sin(\theta)
            \end{equation}
            aus dieser lässt sich die äquivalente Laue Bedingung ableiten, welche aussagt,
            dass ein Röntgenstrahl genau dann gestreut wird, wenn der Beugungsvektor $\vec{k}$ gleich dem
            reziproken Gittervektor ist. 
    
    \subsection{Ordnungsparameter und Phasenübergänge}
            

    \subsection{Überstrukturen}
        
        \subsubsection{$CuZn$}

        \subsubsection{$CuAu$}

        \subsubsection{$Cu_3Au$}

    \subsection{Die röntgenographische Methode}

        \subsubsection{Aufbau eines Röntgendiffraktometers}   

        \subsubsection{Röntgenstrahlung}

        \subsubsection{Intensität der gestreuten Röntgenstrahlung}

        \subsection{Reflexindizierung im Röntgendiffraktogramm}
    
    \subsection{Die resistive Methode}
        


\section{Versuchsafbau}
    \subsection*{Röntgenografische Methode}
    Für die Röntgenografische methode steht im versuch ein Röntgendiffraktometer zur verfügung.
    Dabei sitzt die Probe in der Mitte eines Detektors welcher Röntegenstrahlung detektieren kann.
    Die Probe wird dann mit einer von einer Röntgenröhre erzeugter Röntegenstrahlung bestrahlt, welche
    an den Gitterebenen der Probe reflektiert und auf dem Detektorschirm abgebildet wird.
    \begin{figure}[H]
        \centering
        \begin{subfigure}{.5\textwidth}
        \centering
        \includegraphics[width=.8\linewidth]{images/diffraktometer.png}
        \caption{}
        \label{fig:sub1}
        \end{subfigure}%
        \begin{subfigure}{.5\textwidth}
        \centering
        \includegraphics[width=.6\linewidth]{images/diffraktometer_pic.png}
        \caption{}
        \label{fig:sub2}
        \end{subfigure}
        \caption{a) Schematischer Aufbau eines Röntgendiffraktometers b) Foto des Röntgendiffraktometers}
        \label{fig:test}
    \end{figure}
    \subsection*{Resistives Verfahren}
    Bei dem resistiven Verfahren müssen zunächst die Proben auf einem Stab fixiert werden, anschließen werden
    an der Probe die Leitungen mit Silberpaste leitend befestigt. Anschließend wird der Stab mit der Fixierten
    Probe an einem Schrittmotor Befestigt um die Probe in einen Helium behälter auf variable höhen zu bewegen.
    Beim befestigen der Leiter auf der Probe ist dabei zu achten, dass der abstand zwischen den Kontakten möglichst
    homogen ist und, dass die Kontakte richtig verschaltet sind um die Messung nicht durch die Innenwiderstände
    der Messelektronik zu beeinflussen. Dazu nutzt mann folgenden schaltplan
    \begin{figure}[H]
        \centering
        \begin{subfigure}{.5\textwidth}
        \centering
        \includegraphics[width=.4\linewidth]{images/schaltplan.png}
        \caption{}
        \label{fig:sub11}
        \end{subfigure}%
        \begin{subfigure}{.5\textwidth}
        \centering
        \includegraphics[width=.8\linewidth]{images/aufbaures.png}
        \caption{}
        \label{fig:sub22}
        \end{subfigure} 
        \caption{a) Beschaltung des 4-Punkt messstabes b) Bild des Messaufbaus mit Motor, Proben messtab und Heliumbehälter}
        \label{fig:test1}
    \end{figure}



\section{Durchführung}
    Es werden drei $Cu_3Au$ Proben auf ihre Ordnungsparameter untersucht. Um einen unterschiedlichen Ordnungsparameter
    zu erzeugen, wurden die drei Proben auf über 500 °C ($>T_C = 386 °C$) erhitzt und danach unterschiedlich abgekühlt.
    Um eine vollständige Ordnung zu erzeugen, wird eine der Proben sehr langsam abgekühlt. Für Unordnung wird eine Probe
    in Wasser abgeschreckt. Die teilweise Ordnung wird durch ca. zweistündiges halten bei 370 °C mit anschließendem abkühlen
    erreicht. Von jeder dieser Proben wird nun ein Röntgendiffraktogramm aufgenommen, sowie der elektrische Widerstand 
    mittels der Vierpunktmethode bestimmt. Daraus lassen sich dann die Ordnungsparameter der Proben bestimmen.
 
\section{Auswertung}
    \subsection{Die röntgenographische Methode}
        \subsubsection{Gitterkonstante}
            Die verwendete $K_{\alpha}$-Strahlung beinhaltet zwei verschiedene Wellenlängen ($\lambda_{\alpha 1} = 1.5406 \mathring{A}, \lambda_{\alpha 2} = 1.5444 \mathring{A})$,
            daraus wird eine Wellenlänge
            für die Auswertung gemittelt. Die beiden Strahlungen haben ein Intensitätsverhältnis von $\frac{K_{\alpha 2}}{K_{\alpha 1}} = 0.52$
            \begin{align}
                \lambda = \frac{1 \cdot \lambda_{\alpha 1} + 0.52 \cdot \lambda_{\alpha 2}}{1.52}
            \end{align}
            \begin{align*}
                \lambda = 1.5419 \mathring{A}
            \end{align*}

            Aus der Bragg-Bedingung geht hervor, mit $\Psi =  h^2 + l^2 + k^2$ und $d = \frac{a}{\sqrt{\Psi}}$
            \begin{equation}
                n \lambda = 2dsin(\theta) = \frac{2asin(\theta)}{\sqrt{\Psi}} \Leftrightarrow \frac{n \lambda}{2a} = \frac{sin(\theta)}{\sqrt{\Psi}}
            \end{equation}
            Die linke Seite $\frac{n \lambda}{2a}$ ist eine Konstante, da wir $n=1$ annehmen, die Wellenlänge haben wir oben 
            bestimmt und die Gitterkonstante verändert sich nicht bei einer Probe. Der Fundamentalreflex mit dem kleinsten Winkel
            $2\theta$ entspricht den Indizes (111), da Reflexe $\leq 3$ in diesem Gitter
            verboten sind. Dadurch können wir auch die 
            weiteren Reflexe finden mit der Relation:
            \begin{align}
                \frac{sin^2(\theta_1)}{\Psi_1} = \frac{sin^2(\theta_2)}{\Psi_2} \\
                \Leftrightarrow \Psi_2 = \frac{sin^2(\theta_2)\Psi_1}{sin^2(\theta_1)}
            \end{align}
            \hl{mit} $\Psi_1 = 3$ und die Winkel $\theta_{1,2}$ können aus den Messwerten entnommen werden. Theoretisch
            müsste $\Psi_2 \in \mathbb{N}$, durch Messungenauigkeiten stimmt dies nicht ganz. Daher runden wir $\Psi_2$ immer auf die 
            nächste natürliche Zahl. Dieser Zahl kann nun eine Kombination von Indizes zugeordnet werden, da:
            $\Psi_2 = h^2+k^2+l^2$ die Wahl der Indizes nicht eindeutig ist z.B für $\Psi_2 = 2$, würden (110), (011) und (101) passen.
            
            Um daraus nun die Gitterkonstante a zu bestimmen, wird wieder die Bragg-Bedingung genutzt:
            \begin{equation}
                \Leftrightarrow a = \frac{n \lambda \sqrt{\Psi}}{2 sin(\theta)}
            \end{equation}
            wobei $n=1$, $\lambda = 1.54190 \mathring{A}$, $\Psi$ wie oben beschrieben bestimmt und $2 \theta$ wurde gemessen.
            \begin{table}[H]
                \centering
                \begin{tabular}{c| c| c| c| c}
                    Reflexart & $2 \theta $[\textdegree] & $\Psi_2$ &  mögl. Reflex & a[$\mathring{A}$]\\
                    \hline
                    F & 41.92 & 3 & (111) & 3.733\\
                    F & 48.7 & 4 & (002) & 3.740\\
                    F & 71.36 & 8 & (022) & 3.739\\
                    F & 86.2 & 11 & (113) & 3.742\\
                    F & 91.02 & 12 & (222) & 3.744\\
                    Ü & 23.84 & 1 & (001) & 3.733\\
                    Ü & 33.96 & 2 & (011) & 3.733\\
                    Ü & 54.8 & 5 & (012) & 3.746\\
                    Ü & 60.5 & 6 & (112) & 3.749\\
                    Ü & 76 & 9 & (122) & 3.757\\
                    Ü & 81.56 & 10 & (013) & 3.733\\
                    Ü & 96.26 & 13 & (023) & 3.733\\
                    Ü & 100.76 & 14 & (123) & 3.745\\
                \end{tabular}
                \caption{tab:Gitterkonstanten Probe 2}
            \end{table}
            
            \begin{table}[H]
                \centering
                \centering
                \begin{tabular}{c | c | c | c | c}
                    Reflexart & $2 \theta $[\textdegree] & $\Psi_2$ &  mögl. Reflex & a[$\mathring{A}$]\\
                    \hline
                    F & 41.06 & 3 & (111) & 3.808\\
                    F & 47.86 & 4 & (002) & 3.801\\
                    F & 71.08 & 8 & (022) & 3.751\\
                    F & 85.98 & 11 & (113) & 3.750\\
                    F & 90.5 & 12 & (222) & 3.760\\
                    Ü & 23.5 & 1 & (001) & 3.786\\
                    Ü & 33.74 & 2 & (011) & 3.757\\
                    Ü & 54.6 & 5 & (012) & 3.759\\
                    Ü & 60.3 & 6 & (112) & 3.760\\
                    Ü & 76 & 9 & (122) & 3.757\\
                    Ü & 95.76 & 13 & (023) & 3.748\\
                    Ü & 100.5 & 14 & (123) & 3.752\\
                \end{tabular}
                \caption{tab:Gitterkonstanten Probe 3}
            \end{table}
            
            \begin{table}[H]
                \centering
                \begin{tabular}{c | c | c | c | c}
                    Reflexart & $2 \theta $[\textdegree] & $\Psi_2$ &  mögl. Reflex & a[$\mathring{A}$]\\
                    \hline
                    F & 40.46 & 3 & (111) & 3.862\\
                    F & 47.08 & 4 & (002) & 3.861\\
                    F & 68.88 & 8 & (022) & 3.856\\
                    F & 82.9 & 11 & (113) & 3.863\\
                    F & 87.6 & 12 & (222) & 3.859\\
                \end{tabular}
                \caption{tab:Gitterkonstanten Probe 4}
            \end{table}

            Für jede Probe wird jetzt der Mittelwert der Gitterkonstante mit zugehörigem
            Fehler berechnet, nach den Formeln:
            \begin{equation}
                \bar{a} = \frac{1}{n} \sum^n_i a_i
            \end{equation}
            \begin{equation}
                \Delta \bar{a} = \sqrt{\frac{1}{n(n-1)} \sum^n_i (\bar{a}-a_i)^2}
            \end{equation}  
            es ergibt sich:
            \begin{align*}
                \bar{a}_{Probe2} = (3,74 \pm 0,002)[\mathring{A}]\\
                \bar{a}_{Probe3} = (3,77 \pm 0,006)[\mathring{A}]\\
                \bar{a}_{Probe4} = (3,86 \pm 0,001)[\mathring{A}]
            \end{align*}
            \begin{figure}[H]
                \centering
                \includegraphics[width=0.8\textwidth]{Messdaten/Auswertungsskripte/Probe2.pdf}
                \caption{Röntgendiffraktogramm Probe 2}
                \label{Röntgendiffraktogramm Probe 2}
            \end{figure}
        
            \begin{figure}[H]
                \centering
                \includegraphics[width=0.8\textwidth]{Messdaten/Auswertungsskripte/Probe3.pdf}
                \caption{Röntgendiffraktogramm Probe 3}
                \label{Röntgendiffraktogramm Probe 3}
            \end{figure}
            \begin{figure}[H]
                \centering
                \includegraphics[width=0.8\textwidth]{Messdaten/Auswertungsskripte/Probe4.pdf}
                \caption{Röntgendiffraktogramm Probe 4}
                \label{Röntgendiffraktogramm Probe 4}
            \end{figure}


            \subsubsection{Bestimmung des Ordnungsgrades}
                Um den Ordnungsgrad der Proben zu bestimmen nutzen wir die Formel:
                \begin{equation}
                    S^2 = \frac{I^{\text{Ü}}}{I^F}(\frac{(f_{Au}+3f_{Cu})^F}{(f_{Au}-f_{Cu})^{\text{Ü}}})^2 \frac{(pL_p)^F}{(pL_p)^\text{Ü}}
                    %S^2 = \frac{I^Ü}{I^F} (\frac{(f_{Au} + 3 f_{Cu})^F}{(f_{Au}-f_{Cu})^Ü})^2 \frac{(pL_p)^F)}{(pL_p)^Ü}
                \end{equation}

                %mit dem Lorentz-Polarisationsfaktor L_p \eqref{Lorentz-Polarisationsfaktor}

                Beim bestimmen der Intensität musste darauf geachtet werden, dass das Untergrundrauschen 
                möglichst gut entfernt wird. Dazu haben wir probiert den Untergrund mit einem 
                Fit zu beschreiben, und diesen dann von den Messwerten abzuziehen. Um die Intensität
                nun zu berechnen, haben wir über die Gaußkurven der Peaks integriert. Den Flächenhäufigkeitsfaktor
                kann man aus folgender Tabelle der Anleitung für die möglichen Reflexe ablesen.
                Da man das Verhältnis eines Überstrukturreflexes und einem Fundamentalreflex betrachtet,
                entfällt der zweite Korrekturterm (Absorbtionsfaktor). Die Atomformfaktoren haben wir 
                mit Hilfe der Tablle und Formel aus der Anleitung bestimmt.

                \begin{figure}
                    \centering
                    \includegraphics{images/flächenhäufigkeitsfaktor.PNG}
                    \caption{Flächenhäufigkeitsfaktor}
                \end{figure}

                \begin{table}[H]
                    \centering
                    \begin{tabular}{c | c | c | c | c | c | c}
                        Intensität & $2\theta$ & $f_{Cu}$ & $f_{Au}$ & p & $L_p$ & $S^2$\\
                        \hline
                        199 & 23.84 & 72.66 & 25.93 & 6 & 5.5 & 0.33\\
                        70 & 33.96 & 68.14 & 23.93 & 12 & 2.587 & 0.14\\
                        1868 & 41.92 & 64.66 & 22.36 & 8 & 1.625 & -\\
                        1646 & 48.7 & 61.87 & 21.06 & 6 & 1.159 & -\\
                        89 & 54.8 & 59.53 & 19.93 & 24 & 0.886 & 0.18\\
                        89 & 60.5 & 57.49 & 18.93 & 24 & 0.708 & 0.46\\
                        514 & 71.36 & 53.98 & 17.19 & 12 & 0.499 & -\\
                        69 & 76 & 52.64 & 16.53 & 24 & 0.443 & 0.65\\
                        79 & 81.56 & 51.15 & 15.79 & 24 & 0.395 & 0.66\\
                        814 & 86.2 & 50.00 & 15.22 & 24 & 0.368 & -\\
                        176 & 91.02 & 48.89 & 14.68 & 8 & 0.351 & -\\
                        59 & 96.26 & 47.79 & 14.14 & 24 & 0.342 & 0.87\\
                        74 & 100.76 & 46.91 & 13.72 & 0 & 0.342 & -\\
                    \end{tabular}
                    \caption{Ordnungsgrad Probe 2}
                \end{table}

                \begin{table}[H]
                    \centering
                    \begin{tabular}{c | c | c | c | c | c | c}
                        Intensität & $2\theta$ & $f_{Cu}$ & $f_{Au}$ & p & $L_p$ & $S^2$\\
                        \hline
                        104 & 23.5 & 72.81 & 26 & 6 & 5.668 & 0.137\\
                        50 & 33.74 & 68.23 & 23.97 & 12 & 2.624 & 0.080\\
                        2437 & 41.06 & 65.02 & 22.53 & 8 & 1.702 & - \\
                        1334 & 47.86 & 62.21 & 21.22 & 6 & 1.205 & - \\
                        37 & 54.6 & 59.6 & 19.97 & 24 & 0.89 & 0.094\\
                        40 & 60.3 & 57.56 & 18.96 & 24 & 0.714 & 0.188\\
                        564 & 71.08 & 54.07 & 17.24 & 12 & 0.502 & - \\
                        44 & 76 & 52.64 & 16.53 & 24 & 0.443 & 0.380\\
                        575 & 85.98 & 50.05 & 15.25 & 24 & 0.369 &  -\\
                        89 & 90.5 & 49.01 & 14.74 & 8 & 0.352 &  -\\
                        49 & 95.76 & 47.89 & 14.19 & 24 & 0.342 & 1.445\\
                        79 & 100.5 & 46.96 & 13.75 & 0 & 0.342 & - \\
                    \end{tabular}
                    \caption{Ordnungsgrad Probe 3}
                \end{table}

                Aus diesen Werten wird der mittlere Ordnungsgrad mit Fehler, wie schon bei der Gitterkonstanten, 
                bestimmt:
                \begin{align}
                    \bar{S_2} = 0.47 \pm 0.09\\
                    \bar{S_3} = 0.39 \pm 0.22
                \end{align}

                In den beiden Tabellen gibt es zwei Auffälligkeiten. Bei den Winkeln $\approx 100$°
                ist $\Psi_2 = 14$, welches keinem Flächenhäufigkeitsfaktor zugeordnet werden kann. Daher fallen
                die beiden Werte bei unserer Berechnung raus. Was außerdem auffällt ist, dass bei der Probe 3 
                der Ordnungsgrad des Winkels $95,76$° größer als 1 ist. Da $0 \leq S \leq 1$ sein muss, ist dies 
                vermutlich ein Messfehler.
            


 
\subsection{resistives Verfahren}
    Um in diesem Abschnitt den Ordnungsparameter bestimmen zu können, wurde zunächst der Widerstand der verschiedenen Proben mit
    unterschiedlichem Ordnungsparameter $S$ gemessen. Um daraus den Ordnungsparameter der Proben zu bestimmen
    wurden die Daten zunächst durch eine lineare Abhängigkeit des Widerstands von der Temperatur
    approximiert, welches for $T\Rightarrow 0$ in einen konstanten Restwiderstand übergeht (für die weitere Auswertung wird jedoch lediglich der konstante Bereich betrachtet).
    \begin{figure}[H]
        \centering
        \includegraphics[width=0.8\textwidth]{Messdaten/Auswertungsskripte/Widerstand.pdf}
        \caption{Plot der gemessenen Widerstände in Abhängigkeit von der Temperatur, in rot das gefittete Model ($f(x)=ax+b$ für $T>5K$, $g(x)=c$ für $T<5K$), links von der grauen Linie wird
        ein konstantes Restwiderstandsverhalten angenommen, rechts davon ein lineares Wachstum}
        \label{resistance plot}
    \end{figure}
    Für den Widerstand einer Legierung kann für den Restwiderstand folgendes Verhalten angenommen werden
    \begin{equation}
        \rho_D(x) = \rho_D(T=0) + A\cdot x(1-x)(1-S^2)
    \end{equation}
    wobei $A$ eine zu bestimmende Materialkonstante ist.\\
    Um $A$ zu bestimmen wurde zunächst angenommen, dass aufgrund der Gitterstruktur die Probe mit dem höchsten
    Restwiderstand ebenfalls die größste Unordnung aufweist, dort wurde $S=0$ angenommen. Ebenso wurde angenommen,
    dass die Probe mit dem geringsten Restwiderstand die geordnetste ist. Dort gilt $S=1$. Diese Annahmen sind
    notwendig, um die Materialkonstante aus den Daten zu ermitteln, da sonst unser Gleichungssystem unterbestimmt wäre,
    jedoch sollten alle daraus resultierenden Ergebnisse mit Vorsicht betracht werden, da sowohl ein Ordnungsparameter
    von $\thicksim 1$, als auch von $\thicksim 0$ äussert schwer zu erreichen sind. Nutzt man jedoch diese Annahme, lässt sich die Materialkonstante
    mithilfe der gefitteten Modelle bestimmen. Dazu wird angenommen, dass es sich bei den Proben um $Cu_3Au$ handelt, womit
    aus der Anleitung über $Cu_{1-x}Au_x$ ein Anteil von 75\% Kupfer und 25\% Gold, und damit $x=0.25$ folgt. Damit folgt weiter
    \begin{equation}
        \rho_{D_{max}}(0.25) = \rho_D(T=0) + A\cdot 0.25(1-0.25)(1-S^2)
    \end{equation} 
    $\rho_D(T=0)$ gibt hierbei einen Konstanten Restwiderstand einer reinen Probe bei $0K$ an.
    Ferner folgt
    \begin{align*}
        \rho_{(S=1)} = \rho_D(T=0)\\
        \rho_{(S=0)} = \rho_D(T=0) + Ax(1-x)
    \end{align*}
    \begin{equation}
        A=\frac{\rho_{(S=0)}-\rho_{(S=1)}}{x(1-x)}
    \end{equation}
    \begin{equation}
        \Delta A=\frac{1}{x(1-x)}\sqrt{\Delta \rho_{(S=0)}^2+ \Delta \rho_{(S=1)}^2}
    \end{equation}
    \begin{equation}
        S = \sqrt{\frac{\rho_{(S=0)}-\rho_{(S=?)}}{\rho_{(S=0)}-\rho_{(S=1)}}}
    \end{equation}
    \begin{equation}
        \Delta S = \sqrt{(\Delta \rho_{(S=0)} \frac{\rho_{(S=?)}-\rho_{(S=1)}}{2(\rho_{(S=0)}-\rho_{(S=1)})^2 S})^2 + ( \frac{\Delta \rho_{(S=?)}}{2(\rho_{(S=0)}-\rho_{(S=1)}) S})^2 + (\Delta \rho_{(S=1)} \frac{(\rho_{(S=0)}-\rho_{(S=?)})}{2(\rho_{(S=0)}-\rho_{(S=1)})^2 S})^2}
    \end{equation}
    Um nun S bestimmen zu können muss zunächst aus den Wiederstandswerten zusammen mit den Geometrien
    der Proben der zugehörige spezifische Widerstand bestimmt werden. Dafür gilt folgende Relation:
    \begin{equation}
        \rho = \frac{b\cdot d \cdot R}{l}
    \end{equation}
    \begin{equation}
        \Delta \rho = \sqrt{(\frac{d R \Delta b}{l})^2 + (\frac{b R \Delta d}{l})^2 + (\frac{db\Delta R}{l})^2 + (\frac{db R \Delta l}{l^2})^2}
    \end{equation}
    für die Probengeometrien
    \begin{figure}[H]
        \centering
        \begin{tabular}{l|c|c|c}
             & Länge l [mm] & Breite b [mm] & Dicke d [mm] \\
            \hline
            Probe 2 & $6.7\pm 0.05$ & $5.1\pm 0.03$ & $0.2\pm 0.05$ \\
            Probe 3 & $4.3\pm 0.05$ & $5.2\pm 0.03$ & $0.2\pm 0.05$ \\
            Probe 4 & $4.1\pm 0.05$ & $5.1\pm 0.03$ & $0.2\pm 0.05$ \\
        \end{tabular}
        \caption{Probengeometrien tabellarisch dargestellt}
    \end{figure}
    und damit folgt
    \begin{figure}[H]
        \centering
        \includegraphics[width=0.8\textwidth]{Messdaten/Auswertungsskripte/spezWiderstand.pdf}
        \caption{Plot der Spez Widerstände}
        \label{spec resistance plot}
    \end{figure}
    aus der Modellbetrachtung folgen Restwiderstände von
    \begin{figure}[H]
        \centering
        \begin{tabular}{c|c|c}
            Probe & Widerstand R [$\Omega$] &spezifischer Widerstand [$\Omega m$] \\
            \hline
            Probe 2 & $(3.9\pm 0.0099)e-4 $ & $(5.9\pm 1.5)e-08$ \\
            Probe 3 & $(2.6\pm 0.015)e-4 $ & $(6.3\pm 1.6)e-08$ \\
            Probe 4 & $(2.6\pm 0.00091)e-3 $ & $(6.5\pm 1.6)e-07$ \\
        \end{tabular}
        \caption{Tabellarische Auflistung der Widerstände und spezifischen Widerstände der 3 Proben}
    \end{figure}
    Darauf folgt entsprechend $A=(7.85\pm8.7)e-7 \Omega m$ und damit für Probe 3 auch $S=0.997 \pm 0.018$. Dieses Resultat
    entspricht unseren Erwartungen, da Probe 3 und 2 etwa den selben Restwiderstand bei T=5K besitzen.\\
    Es fällt auf, dass dieses Ergebniss nur für kleine Temperaturen gilt. Würde man diese Auswertung bei
    Raumtemperatur ($\sim 272K$) durchführen würden wir kein verwertbares Ergebnis erwarten.%erwarten, dass der unterschied der Ordnung zwischen Probe
    %3 und 2 größer ist. Gegebenfalls liesen sich die Unterschiede in der Temperaturabhängigkeit durch die lineare
    %Regression berücksichtigen um ein ähnliches Ergebnis zu erlangen, jedoch wäre dieses Ergebnis größer fehlerbehaftet.
        
\section{Diskussion} 

%\begin{itemize}
    \item \textbf{Wie lassen sich die beobachteten Reflexe leicht indizieren?}
    \item {In dem man die verschiedenen Reflexe den Gitterebenen (hkl) zuordnet erhält man mit dem
            Gitterebenenabstand $d_{hkl}=\frac{a}{\sqrt{h^2+k^2+l^2}}$ für kubische Gitter die Netzebenenabstände.
            Daraus folgt mit $n\lambda = 2d_{hkl}sin(\theta_{hkl})$ der Streuwinkel $\theta$, sodass bei 
            bekannter Wellenlänge jedem Winkel/Abstand paar ein millerscher Indize zugeordnet werden kann.}
    \item \textbf{Auslöschungsregeln für $Cu_3Au$}
    \item
    %\begin{align*}
                $$F = \sum_{j=1}^n f_j e^{i \vec{r_i} \vec{G}}$$ %\\
                $$r_0 = (0,0,0), r_1=a(1/2,1/2,0), r_2=a(1/2,0,1/2), r_3=a(0,1/2,1/2)$$% \\
                $$F = c(1+e^{\pi i a (h+k)}+e^{\pi i a (h+l)+e^{\pi i a (k+l)}})$$% \\
                $$\Rightarrow F \neq \text{0 für (h,k,l) alle gerade/ungerade} $$  
    %\end{align*}
    \item \textbf{Absorbtionseffekte sind abhängig von der Probengeometrie weil:}
    \item $$I = I_0 exp(-\mu d)$$
          $$\mu = n \sigma$$
          $\sigma$ = Wirkungsquerschnitt, n = Atome pro Kubikmeter, d = Probendicke
    \item \textbf{Wieso nur benachtbarte Reflexe vergleichen?}
    \item Da $A_T$ Winkelabhängig ist, kann man NUR für benachtbarte Reflexe annehmen das $A_T$
          identisch ist, und damit vernachlässigbar.
    \item \textbf{Widerstand von Kupfer}
    \item \includegraphics[width=0.8\textwidth]{images/copperkek.PNG}
    \item \includegraphics[width=0.8\textwidth]{images/copperwut.PNG}
    \item $$\rho(T) = 1 + \alpha(\frac{T}{\theta}) + cT^5$$
    \item \textbf{Effekt des Linienspektrum auf Messung}
    \item \includegraphics[width=0.8\textwidth]{images/linienspeckie.PNG}
    \item \textbf{Messung der 4 Punkt methode}
    \item \includegraphics[width=0.8\textwidth]{images/4pointstyle.PNG}
    \item \textbf{Wieso umpolen?}
    \item %\includegraphics{}
    \item \textbf{Abhängigkeit vo nder geometrie}
    \item \includegraphics[width=0.8\textwidth]{images/geom.PNG}
    \item \textbf{Wieso umpolen?}
    \item Schottky diode? Maybe? Maybe rauschunterdrückung durch "hohe" Frequenzen.
\end{itemize}

\begin{itemize}
    \item \hl{Haengt $\lambda_{hkl}$ von (hkl) ab? Wenn ja wieso?}
\end{itemize} 
 \section{Anhang}
    \begin{figure}[H]
        \centering
        \includegraphics[width=0.8\textwidth]{Messdaten/Auswertungsskripte/Probe2.pdf}
        \caption{Röntgendiffraktogramm Probe 2}
        \label{Röntgendiffraktogramm Probe 2}
    \end{figure}

    \begin{figure}[H]
        \centering
        \includegraphics[width=0.8\textwidth]{Messdaten/Auswertungsskripte/Probe3.pdf}
        \caption{Röntgendiffraktogramm Probe 3}
        \label{Röntgendiffraktogramm Probe 3}
    \end{figure}
    \begin{figure}[H]
        \centering
        \includegraphics[width=0.8\textwidth]{Messdaten/Auswertungsskripte/Probe4.pdf}
        \caption{Röntgendiffraktogramm Probe 4}
        \label{Röntgendiffraktogramm Probe 4}
    \end{figure}



\end{document}