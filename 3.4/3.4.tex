\documentclass{article}

\usepackage[ngerman]{babel}
\usepackage{pdfpages}
\usepackage{amssymb}
\usepackage{amsmath}
\usepackage{graphics}
\usepackage{graphicx}
\usepackage{geometry}
\usepackage{float}
 
\geometry{
 a4paper,
 total={170mm,257mm},
 left=25mm,
 top=25mm,
}


\begin{document}

    \title{Praktikum B - Versuch B3.4 Positronen-Emissions-Tomografie}
    \date{\today}
    \author{Jesco Talies, Timon Danowski, Erik Gasmus}
    \maketitle
    \newpage

    \tableofcontents
    \newpage
    
    \section{Einleitung}
        \subsection{Versuchsaufbau}

        \subsection{Ziel}
        Die Positronen-Emissions-Tomografie macht sich die Eigenschaft der $e^+e^-$-Vernichtung zu nutze. Die entstehenden Photonen werden dabei mit zwei Detektoren Koinzidenz nachgewiesen werden.
        Der Ziel des Versuches ist der Nachweis von Radioaktiven Quellen in einem Verschlossenen Behälter und die Untersuchung der relativen Intensitäten der Quellen und die Winkelabhängigkeit.

    \section{Versuchsvorbereitung}
        Für das Verständnis:
        \begin{enumerate}
            \item Warum PET?
            \begin{itemize}
                \item Um Veränderungen in Gewebe schon während der Veränderung zu erkennen werden abbaubare radioaktive Tracer injiziert deren Aktivität durch eine Koinzidenzmessung bestimmt wird.
                Bei einer Koinzidenzmessung zerstrahlt ein Positon mit einem Elektron zu zwei gamma quanten mit gleicher Energie E=511 keV die Kollinear auseinander fliegen. Wir in zwei Detektoren 
                gleichzeitig ein Ereignis registriert nennt man das Koinzidenz und diese Koinzidenz ist sehr wahrscheinlich auf die Zerstrahlung eines Gamma-Quants zurückzuführen.
                Aus eine großen Anzahl dieser Aktivität lässt sich die Aktivität in dieser ebene rekonstruieren.
                Die Tracer müssen eine geeignete Halbwertszeit haben und werden daher i.d.R. in einem Synchrotron in der nähe des PET hergestellt.
                Es kommen nur Kerne mit \beta +- Zerfall in Frage.
                Es kommt zu Falschmessungen durch Positronen die nicht am or ihrer Emission zerstrahlen sondern ihre 2mm Reichweite im Gewebe nutzen
            \end{itemize}
            
            \item Ziel des Versuchs
            \begin{itemize}
                \item Ziel ist es die physikalischen und elektronischen Grundlagen der PET kennenzulernen.
            \end{itemize}

            \item $\beta$-zerfall
            \begin{itemize}
                \item Der Beta+ Zerfall ist ein Radioaktiver Zerfallsprozess bei dem ein Proton in ein Neutron umgewandelt wird, dabei wird ein Positron und ein Neutrino erzeugt.
                die Massenzahl A bleibt konstant
                Der äquivalente Prozess zum \beta +- Zerfall ist der Elektroneinfang Prozess , dort entstehen jedoch keine Positronen
            \end{itemize}

            \item Paarvernichtung
            \begin{itemize}
                \item eine entstandenes Positron wird in der Materie zunächst abgebremst und kann danach mit einem Elektron unter Aussendung zweier gamma Quanten zerstrahlen.
                Wenn sich das Elektron-Positron paar in ruhe befindet werden aufgrund der Impulserhaltung die beiden abgestrahlten Photonen kollinear d.h. im 180° Winkel abgestrahlt.
                Der Wirkungsquerschnitt ist stark Energieabhängig
            \end{itemize}

            \item Szintillatoren und Photonmultiplier
            \begin{itemize}
                \item Ein Anorganischer Szintillationszähler besteht aus einem Szintillationskristall und einem Photomultiplier. Im Kristall werden über Anregung und Photonemission
                Ereignisse detektiert. Austretende Photonen werden dann durch den Photoeffekt in Freie elektronen umgesetzt und im Photomultiplier beschleunigt und vervielfältigt.
                Die Stärke des Signal ist direkt proportional zur Energie des Ereignisses
            \end{itemize}
        \end{enumerate}



        Folgende Fragen sind notwendig für das Verständnis des Versuchs und dienen als Vorbereitung für diesen.
        \begin{enumerate}
            
            
            \item Warum PET?
            \begin{itemize}
                \item Was ist grundlegend nötig um PET zu betreiben?
                    Eine Quelle für geeignete Isotope (zb. einen Synchrotron) als Tracer. Und einen Detektor zur Messung der Koinzidenzen.
                \item Welche Eigenschaften sollte ein Isotop haben um in frage zu kommen?
                    Eine relativ kurze Halbwertszeit im Stunden Bereich und einen $\beta^+$ zerfall
                \item Was genau Versteht man unter falschen Koinzidenzen?
                    Es entstehen und verschwinden stets Koinzidenzen auf natürliche Art und Weise, so gibt es Umweltstrahlung, Streuung der künstlichen inzedenzen oder Absorption.
            \end{itemize}

            \item Ziel des Versuchs
            \begin{itemize}
                \item Wie muss die Truhe gescannt werden um ein eindeutiges Ergebnis bzgl. der Lage und der Aktivität der quellen zu erhalten?
                Abschnitt 5.6
            \end{itemize}

            \item $\beta$-zerfall
            \begin{itemize}
                \item $\beta^+$-Zerfall
                    Beim $\beta+$ zerfällt ein Proton in ein Neutron unter Entstehung eines Positrons und eines Neutrinos
                    $$(Z,N) \rightarrow (Z-1,N+1) + e^+ + \nu_e$$
                \item $\beta^-$-Zerfall
                    Beim $\beta-$ zerfall zerfällt ein Neutron unter Entstehung eines Elektrons und eines Antineutrinos in ein Proton
                    $$(Z,N) \rightarrow (Z+1,N-1) + e^- + \bar{\nu_e}$$
                \item Elektroneneinfang
                    Beim Elektroneneinfang nimmt ein Kern ein freies Elektron auf und wandelt unter Entstehung eines Neutrinos ein Proton in ein Neutron um
                    $$(Z,N) + e^- \rightarrow (Z-1,N+1) + \nu_e$$
                \item Wo auf der Nuklidkarte findet man $\beta^+$ bzw. $\beta^-$ Strahler?
                    $\beta^+$ Findet man bei Kernen mit einer Höheren Protonenzahl Z als ihr Stabiler Kern und \newline
                    $\beta^-$ bei Kernen mit einer niedrigeren Protonenzahl
                    \begin{figure}[h]
                        \centering
                        \includegraphics[scale=0.3]{Images/1280px-Nuklidkarte_Segre.svg.png}
                        \caption{Nuklidkarte}
                        \label{fig:Nuklidkarte}
                    \end{figure}
                \item Zerfall von $^{22}Na)$
                    $^{22}Na$ Zerfällt über $\beta^+$ zerfall in das Stabile Element $^{22}Ne$ bei einer Energieemission von 2.842MeV.
                    Es ist ein Standard Tracer der in der Medizin genutzt wird. Es besitzt eine Halbwertszeit von 2 Jahren und 220 Tagen.
            \end{itemize}

            \item Paarvernichtung
            \begin{itemize}
                \item Unter welchen Voraussetzungen kommt es zur Annihilation?
                    Es wird ein Positron mit geringer Kinetischer Energie in Materie benötigt welches dann mit einem Elektron ein Positronium Atom bilden kann
                    und im Fall das der Spin des Positrons dem des Elektrons entgegengesetzt ist zerstrahlen beide.
                \item Welchen Einfluss hat die kinetische Energie des Positrons auf die Annihilation?
                    Je mehr Kinetische Energie das Positron einbringt, desto stärker weichen die Austrittswinkel von 180° ab. Diese Abweichung lässt sich beschreiben über
                    $$ tan(\theta) = \frac{p_T}{m_ec}$$
                    Wobei $\theta$ die Abweichung von 180° und $p_T$ die transversalkomponente des Positronium Impulses gegenüber der Emissionsrichtung angibt
                \item Welchen Einfluss haben die Restenergien bzw. -Impulse auf die enstehenden $\gamma$-Quanten?
                    Die Restenergien und Impulse werden auf die entstehenden gamma Quanten aufgeteilt und ihnen übertragen
            \end{itemize}

            \item Szintillatoren und Photomultiplier
            \begin{itemize}
                \item Was sind die Vor- und Nachteile von anorganischen Szintillationsdetektoren?
                    Vorteile:
                    \begin{itemize}
                        \item Schnelle Ansprechzeit (einige ns)
                        \item Sensitiv auf deponierte Energie
                        \item (Herstellung und Betrieb unkompliziert) 
                    \end{itemize}
                    Nachteile:
                    \begin{itemize}
                        \item Empfindlich gegenüber Magnetfeldern
                        \item 
                    \end{itemize}
                \item Wann spricht man von Fluoreszenz und wann von Phosphoreszenz?
                    Floureszenz ist die Emission eines Photons mit größerer oder gleicher Wellenlänge innerhalb von wenigen Nanosekunden \newline
                    Phosphoreszenz hingegen kommen die entstehenden $e^-$ zuerst in einen metastabilen zustand und werden erst nach einigen Minuten detektiert
                \item Warum ist es nötig den Photonmultiplier abzuschirmen?

            \end{itemize}
        \end{enumerate}
    \section{Versuchsdurchführung}
        Messaufträge:
        \begin{enumerate}
            \item Vorverstärkersignal am Oszilloskop. Detektoren and Hochspannung anschließen, quelle in der Mitte der Detektorfläche platzieren. Signal skizzieren.  
            \item TFAs einbauen und das Gaussignal am Oszilloskop betrachten
            \item CDDs einbauen und die Nulldurchgänge am Oszilloskop untersuchen, verschiedene einstellungen des CFD testen, Einfluss auf die Zählrate beobachten
            \item Ortsauflösung bestimmen, Lage der quelle bezüglich der Line of Response ändern, dazu Quelle auf wagen legen, wagen mit fester Schrittweite verschieben
            im vom Detektor abgedeckten Bereich, Anzahl der Koinzidenzen in 60s Messbereich messen, insgesamt 20 Messwerte
            \item Messen der Zählraten auf X Achse -> Y Achse und dann diagonal 
            \item Winkelabhängigkeit: CFD so einstellen das nur die 511 keV Linien betrachtet werden, Winkelbereich -5° - 5° in 0.5° schritten messen.
            Danach 511keV auf dem 1ten Detektor messen und 1275 keV Linie auf dem 2ten Detektor messen. Aufgrund der Mittleren Lebensdauer von 22Ne von 4 pikoSekunden und der
            hohen Wahrscheinlichkeit eines angeregten Kerns ist wahrscheinlich keine signifikante reduktion der Zählrate zu erwarten
        \end{enumerate}
    \section{Messwerte}

    \section{Auswertung}
        \begin{itemize}
            \item Höheres Signal 1275er Linie, niedrigeres 511keV
            \item Signalhöhe -> Signalstärke/Energie, Signalbreite -> TFA bedingt / FWHM
            \item TFA -> Gausskurve -> CFD Nulldurchgänge -> Coincidence Counter Logic1/0
            \item 
        \end{itemize}

        \subsection{Signale Am Oszilloskop}
            Im ersten Abschnitt der Auswertung befassen wir uns mit der Signalverarbeitung und den verschiedenen
            Schritten, in dem wir die verschiedenen Ausgangssignale der elektronischen Signalverarbeiter am Oszilloskop betrachten.


            \subsubsection{Szintillator und Vorverstärker}
                Folgende Graphik zeigt das Signal direkt nach der Szintillatorschaltung, bestehend aus
                dem Szintillator selbst, dem Photomultiplier und dem Vorverstärker. 
                \begin{figure}[H]
                    \centering
                    \includegraphics{Images/SzintillatorSignal.JPG}
                    \caption{Szintillator Signal}
                \end{figure}
                Obwohl man nach physikalisch ein Negativen Spannungsimpuls erwarten würde ist dieses Signal jedoch Positiv, dies ist für die
                Messung jedoch vollkommen unerheblich und ist lediglich durch die Wahl der Vorverstärkers bestimmt. 
                Die Höhe des Peaks (hier ca. $390mV$) ist dabei direkt proportional zur Energie des gemessenen
                Ereignisses. Die Breite des Peaks (hier ca $200\mu s$) entsteht hierbei hauptsächlich durch das entladen von Kapazitäten
                welche im Vorverstärker verbaut sind.
                
            \subsubsection{Timing Filter Amplifier}
                Das folgende Signal ist der Ausgang des TFA bei Eingang des obigen Signals der Szintillatorschaltung.
                \begin{figure}[H]
                    \centering
                    \includegraphics{Images/TFASignal.JPG}
                    \caption{TFA Singal}
                \end{figure}
                Auch hier fällt direkt auf, das die Polarität des Spannungsimpulses gewechselt hat, auch dies ist für die Auswertung irrelevant und
                nur durch die Wahl des TFA bestimmt. \newline
                Das Signal wird durch den TFA in seinem Informationsgehalt kaum beeinflusst, er dient lediglich der Aufbereitung für
                die Verarbeitung durch den CFD. Die Höhe des Peaks (hier ca. $4,85V$) und Breite des Signals (hier ca. $1s$) tragen dabei die selben
                Informationen.

            \subsubsection{Constant Fraction Discriminator}
                Nun kommen wir zum ersten auswertenden Element, die folgende Signale sind sowohl der Analoge- als auch der Digitale des CFD.
                \begin{figure}[H]
                    \centering
                    \includegraphics{Images/CFDSignal.JPG}
                    \caption{CFD Signal, Analog(Gelb) Digital(Blau)}
                \end{figure}
                Das analoge Signal trägt nun nicht länger unsere Informationen, es dient an dieser stelle lediglich der Veranschaulichung des
                Prinzips des CFD, da man deutlich erkennt, das dieser bei Nulldurchgang des Verarbeiteten Signals ein Logisches Signal
                von ~0.8V bzw. Logisch 1 emittiert, welches für ~1s bestehen bleibt. Die Signalhöhe ist dabei eigentlich egal, da uns hierbei nur
                dessen Logisches Äquivalent interessiert, die Signalzeit hingegen ist äußerst interessant, da diese unseren Toleranzbereich definiert
                indem 2 Logische Signale noch als Koinzidenz anzusehen sind. Welche Signale überhaupt einen Logischen Output erzeugen lässt sich über 
                die Obere- und Untere Schwelle einstellen, durch die Signale mit Höhen unter oder Über den Grenzwerten verworfen werden. Diese Einstellungen
                sind notwendig um sowohl schwaches Hintergrundrauschen unterhalb der unteren schwelle zu verwerfen, als auch die 1275keV Linie zu verwerfen,
                da für die Gewünschten Zählraten nur Teilchen mit Energien um 511keV gemessen werden sollen.
            
                \subsubsection{Zusammenfassung}
                Um den oben Dargestellten Prozess noch einmal kompakt zusammenzufassen:
                \begin{itemize}
                    \item Der Szintillator liefert einen Peak proportional zur Teilchenenergie des Detektierten Teilchens
                    \item Im TFA wird das Signal weiter aufbereitet und Verstärkt
                    \item Der CFD Bestimmt ob es sich um einen zulässigen Peak handelt und emittiert ein entsprechendes Logisches Signal
                \end{itemize}
        
        \subsection{Bestimmung der Ortsauflösung}
            Um nun die Ortsauflösung unseres Aufbaus zu bestimmen wurde die Anzahl der detektierten Ereignisse in 60 Sekunden
            für verschiedene Verschiebungen der Quelle relativ zum Aufbau aufgetragen. Die daraus resultierende Verteilung wird gut genähert durch eine
            Gaußkurve wie im folgenden Bild dargestellt wird:
            \begin{figure}[H]
                \centering
                \includegraphics[width=0.9\textwidth]{Ortsauflösung.png}
                \label{Plot der Ortsauflösungsmessung}
            \end{figure}
             

        \subsection{Auswertung des PET-Scans}
        
        \subsection{Analyse der Winkelabhängigkeit}


    \section{Diskussion}

\end{document}